\section{What is Engagement: A Behavioural Perspective}
\label{engagement}
Playing games has always been present in human history as an occupation aiming to entertain and relax [42], it can be defined as a free-time activity with spatial and temporal boundaries able to intensely absorb who is involved in it [42]. A special case of the broader group of games are those being delivered and experienced in a digital format (i.e. videogames) which since various decades has been substituting more traditional ludic activities [42, 43]. This phenomenon has been well reflected both in terms of number of people involved in playing videogames as well as in the amount of time spent engaging in this activity [44]. One of the main reasons for this explosive phenomenon relies on the fact that videogames seems to be perfect medium for delivering enjoyable experiences [32], consequently holding a strong potential to engage and retain users involved in the playing activity. Various attempt has been made to understand engagement in videogames, both at the level of process and factors driving and influencing it [32].

\subsection{Factors Driving Engagement}
\label{factors_engagement}
The literature reports various theoretical approaches  for addressing the possible different mechanism producing engagement [32], we will briefly illustrates some of the most prominent ones focusing in a second moment on a specific framework connecting engagement to the motivational factors provided by videogames. Focusing on the motivational factors driving engagement is a well suited ground for building a methodology aiming to assess and evaluate engagement via behavioural measures, this because we can take advantage of existing theories describing human behaviour via the interaction between the individual (i.e. the player) and the stimuli (i.e. the game features) present in the environment (i.e. the game world).

Flow
A classical construct employed in the videogame literature for explaining the phenomenon of engagement, is the concept of flow developed by [33]. The formulation of this construct prescribe that when an individual is absorbed in an activity perceived as valuable they will experience a rewarding state of optimal pleasure constituting the fuel for of  engagement process [32]. In this view the conditio sine qua non for the flow state to arise is a balanced combination of the individual’ skill level and the difficulty of the task in which they are involved. Despite offering an interesting point of view, the concept of flow as a framework for explaining engagement in digital games might be prone to the fallacy of circular reasoning: is a user engaged in a specific activity because this provides the optimal flow experience or this last one is a bi-product of being engaged in the activity itself? 

Immersion
A construct linked to the concept of flow is immersion, which differently from flow is concerned with the specific, psychological experience of engaging with a computer game [46]. Immersion is referred as the experience of engaging in one moment in time with a videogame rather than being a factor driving engagement itself [46]. The experience of immersion involves loosing track of time, space and having a sense of being in the task environment, all of this occurring as a result of a good gaming experience. What emerges from the brief overview of the concept of immersion is that rather than posing itself as a factor influencing or driving engagement it seems to provide an alternative approach for describing and characterizing it.

Uses and gratifications
Uses and gratification theory states that users possess different motives characterizing their gratification-seeking behaviours and that these can be satisfied through media consumption [47]. Engagement in this type of activities (i.e. media consumption) is justified by their ability to meet the motives driving the user behaviour. Another point that this theoretical approach assumes is that media audience are not passive but rather variably active, concept that is particularly relevant when evaluating the interaction between players and videogames, since the first have an active role in determining how the latter changes over time[47]. Despite its relatively vague formulation, uses and gratification theory introduce two important concepts: first that users engage in a spontaneous activity (i.e. media interaction) in search of some form of gratification and second that this interaction is not passive but an active process in which the user interact with the medium.

\subsection{The Engagement Process Model}
\label{eng_proc_model}
Various attempts have been made for formalizing the concept of ‘engagement in digital games’, but this appears to be a non-trivial task with the resulting theoretical framework being heterogeneous and not unified [32]. We will employ the work of [40] for providing a general overview on engagement. In [40] engagement is defined as:
‘…a quality of user experiences with technology that is characterized by challenge, aesthetic and sensory appeal, feedback, novelty, interactivity, perceived control and time, awareness, motivation, interest and affect’.
Building on this, [40] avoids to provide an exact definition of engagement but rather describes it as a process with distinct phases each one possessing peculiar attributes.
Point of engagement
It Is the starting point of the engagement process, it configures as the moment in which the user’s attention is directed towards a specific activity due to properties of the activity itself or for its capacity to fulfil specific motivational drives that the user might have.
Period of engagement
It is the period during which the user is having a sustained involvement into the activity and it is fostered by the ability of the activity to provide appropriated feedbacks, novel information and features.
Disengagement
It is defined as the moment in which the users makes an internal decision to stop the activity or when external factors force them to not engage in it anymore. The internal factors are usually connected to loss of interest or pressure derived by the time passing, external factors instead relate to the inability of the activity to provide novelty or to the occurrence of distractors during the activity.
Reengagement
It identifies the moment in which the user returns to the activity after disengagement occurred. This can happen both at short term and long term and it is the result of positive past experiences with the activity, which are usually linked to be exposed to rewarding incentives or novel content within the activity.
What emerges from the description of engagement as process seems to be that a mix of properties of the activity, internal states of the user and ‘environmental’ factors external to the activity itself controls the engagement’s quality and quantity. This consideration will be functional for the theoretical framework we are trying to build through this review.

SUMMARY
The aim of this brief review was not to define or once-for-all clarifying the concept of engagement in digital games (it would be almost impossible other than out of scope) but rather to identify a possible fil rouge connecting the existing works. What emerged seems to point in the direction of engagement being a second order factor emerging from (or generated by) a series of subjective experiences linked to the interaction between the user and the playing activity. What seems to connect all these different approaches seems to be the description of a process in which the interaction between the user and the game activity generates feelings of enjoyment and pleasure ultimately contributing to the engagement in the activity itself [32]. This interpretation will become particularly useful when we will address the concept of reward and incentive salience later on.

\section{Motivation}
\label{motivation}

The construct of motivation is a key concept for understanding why individuals seek out specific objects or experiences at particular times and why they react in particular ways when encountering stimuli considered of particular relevance [1]. In this view motivation can be defined as the process leading the modulation and reiteration of goal directed behaviours that once reached exerts rewarding (i.e. pleasurable) effects on the individual [2]. What is a common confound when defining this phenomenon is to partition it in different sub-components each one supposedly representing a different version of the motivational process (e.g. the motivation of X, the X motivation) [3]. The motivational process remains always the same, what changes is the nature of the goal producing the motivated behaviour [2]. If we represent motivation as a vector, its length would be the intensity of pursuit (or the amount of goal directed behaviour) while the angle the focus on a specific goal [2]. Summarizing we can say that the behaviour of an individual is motivated by the expectancy of pleasurable outcomes derived by the goal the behaviour is aiming to reach [1]. The aforementioned statement becomes particularly relevant for those spontaneous activities which are not driven by the fulfilment of fundamental needs (e.g. hunger or thirst) or by the avoidance of negative consequences (e.g. cognitive or physical pain), indeed for these activities what matters is the nature of the goal which may largely vary among individuals. Having a clear definition of the motivational process in place will help us understanding how some of the findings regarding motivational factors in videogames can be reframed in a behavioural context.

\section{An Historical View on Reward-driven Motivation}
\label{motivation_hist}
One of the early theorizations of human motivation proposed that individuals were motivated by the expectance of incentive outcomes. These expectations are formed through a process in which an individual learn the existence of an association between the actions he performs and the potential pleasurable outcomes associated to them [1]. Extending on this conceptualization, it has been proposed that the learning process does not only provide a way for forming the expectation of a pleasurable outcome in response to a specific behaviour but also allow to perceive the behaviour itself as source of hedonic reward [1], this concept of hedonic reward will be clarified later in this section as it constitutes an interesting point of connection with existing models of motivational factors in videogames. A third theoretical formulation, which fuses the two aforementioned approaches, making use of  concept related to learning through reinforcement formulated that a stimulus and the behaviours associated to it become relevant and salient for an individual as a consequence of learning its rewarding properties [1]. What written in this paragraph seems very distant from our initial starting point, to a certain extent an over-simplistic lab rat view on human behaviour, but what has been described here is one of the fundamental and most elegant (in terms of complexity to explanatory power ratio) mechanism able to explain and describe at why people do what they do, why they keep doing it and why they do that for a specific amount of time. Obviously taking this as the ultimate explanatory approach would be not just naïve but wrong as many layers of complexity need to be applied for approaching a good description of the motivational process in humans, however we can see some similarities with our previous overview on engagement: the relationships between the individual and the activity preformed, the ability of the activity in providing pleasurable experience and the necessity of feedback (i.e. reward) for maintaining engagement.

\subsection{Reward and Incentive Salience Attribution}
\label{incentive_salience}
Until now we have mainly used the terms reinforces and incentives for identifying stimuli able to drive and shape behaviour, but when it comes to define effective reinforces, is not just a matter of merely pairing a behaviour with a stimulus but the stimulus itself has to have particular properties. In this view, stimuli able to generate pleasurable feelings in the individual are the best candidates for being effective reinforces,  they are said having ‘rewarding properties’.  But what is, and how can be defined the reward?  The reward is a process generated in response to a stimulus making it desirable for its capacity to generate pleasurable responses. In this view, for being able to generate rewarding response, a stimulus needs two fundamental properties: it has to be wanted (i.e. it acquires the capacity to become desirable) and liked (i.e. it has to be able to generate pleasure in the individual) [37]. But how a particular object acquires this properties? Here comes in action the aforementioned learning processes: the repeated pairing between a stimulus and the reaction it generates in the individual imbue the first with rewarding properties. Moreover, through the aforementioned learning process (i.e. operant conditioning) not just the stimulus itself but also the connected instrumental behaviour will likely acquire the same rewarding properties [37]. A useful distinction that can be me here is between stimuli having primary and secondary reward properties:
Primary rewards 
These are type of stimuli having intrinsic rewarding properties due to their linkage with essential evolutionary needs (i.e. satisfaction of homeostatic needs),  the rewarding properties don’t have to be learned but already present, up to a certain extent, in the stimulus itself [45]. 
Secondary rewards
these are type of stimuli which don’t hold an innate capacity to generate rewarding experiences, their capacity to generate pleasurable experiences and consequently being wanted is strongly related to the occurrence of some of the aforementioned learning processes [45].

Despite this distinction might seems superfluous at a first glance, it has a particular relevance for understating the framework in which videogames lies. No element in a videogame context possesses intrinsic rewarding properties, here the learning process is essential for allowing the element to acquire the aforementioned rewarding properties. The attractive property of an in-game element is not present before the user start playing the game but it might be learned through the user – game interactions occurring during the playing activity.

\paragraph{Wanting}
\label{wanting}
\lorem

\paragraph{Liking}
\label{liking}
\lorem


\paragraph{Learning}
\label{learning}
Classical conditioning describes a learning process in which, independently from the activity of an individual, the repeated pairing of two stimuli will cause one to acquire the eliciting properties of the other [36]. Employing the aforementioned concepts of reinforce, the repeated pairing of a neutral object with reinforcing consequences will imbue the first with reinforcement properties. This concept is extremely simplistic and not suitable for describing human behaviour, however succinctly illustrate a very basic and common learning process other than being the ground from which operant conditioning stems.

Operant conditioning extends on the concept of classical condition introducing the agency of the individual. The concept refers to the process in which the frequency of a behaviour tends to increase when precise consequences are associated to it [35]. In this view, an operant is formalized as a goal directed behaviour while all the elements reinforcing the re-iteration of this behaviour are called reinforces [35]. The learning process here results from the relationship between a behaviour and its consequences, therefore the probability of behaviour to take place is related to its capability to produce reinforcement [34]. Again, reframing this concept in a videogame context the goal directed behaviour would be the interaction with the game environment while the reinforces are the consequences generate by the interaction itself, which might come in form of internal state only (i.e. pleasurable experiences) or also in the form of in game rewards (i.e. game feedbacks).

We will now illustrate the concepts of classical and operant conditioning, basic learning processes which are fundamental for introducing the higher level concept of incentive salience later on. For understanding the processes of classical and operant conditioning it is worth defining what a reinforcing object and process are. We can describe as reinforces those object having the capability to alter the probability of appearance of a specific behaviour [34, 35, 36].  On the other hand the reinforcing process identify the development of a condition in which a specific behaviour becomes more probable when followed by particular reinforcing events and becomes less probable when this last ones are not present anymore [34]. Reframing this concepts in a videogame context we can hypothesize that in-game contents or mechanics act as reinforcing objects when they produce positive outcomes for the player, in a cyclic process in which interacting with the game might produce positive reactions player-activity interactions which in response will make new interactions with the game more probable. 
We will now see the how classical and operant conditioning operationalize the aforementioned concepts of reinforcing object and process.

\section{Motivation and Engagement}
Motivation and engagement seems to be interchangeable terms but we argue that they are qualitatively different. Motivation pertain the state of an individual with resect to achieving a particular goal, perfroming a particular action or interacting with a particular object. Engagement on the other hand describes the quantity and quality of the interaction both from a behavioural and experiential point of view. From now on the relationship between motivation and engagement will be assumed we will have to rember that engagement, from a behavioural point of view, is a derivative product of the motivational state in which an individual is. 

\section{Methodologies for Measuring Motivation}
    \subsection{Self-report Measures}
    \subsection{Behavioural Measures}
    \subsection{Psychophisiological Measures}

\section{Estimating motivation and engagement at scale from observational Data}
\lorem

\lorem


