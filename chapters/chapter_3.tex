\section{Introduction}
\label{implementation_testing_introduction}

In this chapter we will outline the implementation of the model described in chapter 2. We adopted a variation of bottom-up iterative model building process in which first the simplest version of a model is designed, built and tested and then, based on the perfromance of this model and addition theoretical assumption a new improved version of the same model is designed. At each stage of the model building process we test the new version of the model against alternative approaches in order to test for specific hypothesis stated in the model design stage. Each section of this chapter corresponds to a different version of the model and will have the following structure. In Model design we state the task the model is trying to solve, the design of such model and the theoretical assumptions that informed the design. In data we describe the dataset used for evaluating the perfromance of our model along with any data-related processing. In model Comparison Procedure we outline the alternative models against which our model is compared and the procedure adopted for the comparison. In results we describe the outcome of the model comparison procedure. In model criticism we discuss the results in light of the theoretical assumptions stated in Model Design and highlight potential improvements of the model design.

\section{Joint Prediction of Future Behavioural Intensity}
\label{model_architecture_1}

\subsection{Model Design}
We present a novel deep neural network architecture, loosely inspired by the winning entry in \cite{lee2018game}, for jointly estimate survival time and churn probability. This architecture, the `Bifurcating Model' (BM), is demonstrated in Fig. \ref{bm_architecture}. The model receives, as input, both a vector of unfolded features, as in Table \ref{metricsdescription}, as well as a context vector containing a numerical encoding of the game (e.g. jc3 = $[1]$, lis = $[2]$ etc.). The game context is then embedded into a vector of $l = 40$, similarly to what is done in words embedding for sentiment analysis \cite{chollet2015keras}. Differently from a one-hot encoding, this approach provides a non-sparse representation of the input while also projecting it into a multi-dimensional space where the relationships between elements become meaningful (e.g. game contexts which are similar to each other in respect to the objectives will be located closer to each other in the embedding space). Using an embedding for encoding the game contexts allows to have a representation that grows richer and richer the more categories are included into it. Obviously this would require to re-train the model whenever a new unseen context is added, practice however not just advisable but also routinely done in production. Next, the raw behavioural input and the embedded game context vector are concatenated along the temporal dimension into a single feature vector and a zero-padding re-applied where needed. At this point, a masking layer allows the model to more efficiently work with time-series of different lengths (i.e. skipping the computations for the zero-padded time-steps) and a dense layer, applied to each time step, to combine raw behavioural metrics and context in a new vector of $l = 40$. These newly obtained features are then modelled across time using a Long Short-Term Memory (LSTM) recurrent layer with $n = 100$ units. Therefore, the output of this LSTM Layer is a feature vector of $l = 100$ which is a latent representation of the input features across time and can be seen as providing a high-level representation of the behavioural state of the user during the OP. The final step of this architecture is to then take this high-level latent representation and pass it to a pair of shallow NNs, one tasked with estimating survival time and the other churn probability. These estimators are formed of a pair of densely connected layers, where the first layer has $n = 300$ units and the last has $n = 1$ units, the output of which will constitute the survival time and churn probability estimates. Like the two MLP models the BM was batch trained with a batch size of 256 until convergence using the ADAM optimizer, with learning rate adjusted through a cyclical policy \cite{smith2017cyclical, chollet2015keras}, minimizing the sum of the two losses. Similarly to the MLP models, the hidden layers used $ReLU$ as activation function whereas the two outputs units used respectively an $identity$ and $sigmoid$ functions for producing the survival time and churn probability estimates. For the survival time branch SMAPE was used as an objective function while for the churn estimation branch BCE was adopted. We applied two regularization techniques after the computations of the first layers of each shallow NN, batch normalization \cite{ioffe2015batch} and dropout \cite{srivastava2014dropout} ($rate = 0.1$). Additionally, following the intuition from \cite{gal2016dropout}, we employed dropout also at inference time for sampling from the model parameters and obtaining a distribution over the posterior so to be able to represent uncertainty in the model estimates. This was achieved by querying the model 50 times at prediction time and retaining all the produced values. When computing the performance metrics we then used the mean of the estimated values, since they roughly followed a normal distribution the mean could be seen as the value with highest probability. All the experiments were implemented in Python 3.6, with the algorithms for Experiment 1 and 2 provided by the library scikit-learn \cite{scikit-learn} and our novel BM architecture developed using Keras with Tensorflow as a back-end \cite{chollet2015keras}.

\subsection{Data}
To conduct our experiments, we gathered data from six different games published by our partner company, \textit{Square Enix Limited}. Focusing on maintaining heterogeneity in genre and platform, we considered the following titles: \emph{Hitman Go} (hmg), \emph{Hitman Sniper} (hms), \emph{Just Cause 3} (jc3), \emph{Just Cause 4} (jc4), \emph{Life is Strange} (lis), and \emph{Life is Strange: Before the Storm} (lisbf). A general description of each of these titles can be found in Table \ref{gamesdescription}. Data were gathered from any user playing between the game's release and February 2019, allowing us to adopt more robust sampling strategies which utilizes the breadth of virtually the entire user-base. To rule out possible `faulty' but not `naturally abnormal' data, we restricted the data cleaning process to a single filter applied at query time to ignore users having at least one of the considered metric over the game population's \nth{99} percentile. This allowed us to make little assumptions on the distribution of the data as well as providing a convenient stress test for eventual future applications.

\begin{table}[h] \centering
\tabcolsep=1pt\relax
\caption{\textbf{Data-set Description}. For each game we retrieved 80,000 Churners and 80,000 Non-Churners randomly sampled from all the available users.}
\label{gamesdescription}
\begin{tabular}{@{}lrrrrrrl@{}}
\toprule

\multirow{2}{*}{\textbf{Game}} & \multicolumn{2}{l}{\textbf{Survival Time (Mins})} & \multirow{2}{*}{\textbf{Churners}} & \multirow{2}{*}{\textbf{Non Churners}} & \multicolumn{2}{l}{\textbf{Observation Period}} & \multirow{2}{*}{\textbf{Descriptive Tags}} \\ \cmidrule(lr){2-3} \cmidrule(lr){6-7}
                      & \textbf{Min}                  & \textbf{Max}                  &                           &                               & \textbf{Min}                & \textbf{Max}               &                                \\ \midrule
hmg                        & 11 & 260    & 80,000 & 80,000  & 1  & 7  & Mobile, Single Player, Strategy                       \\
hms                        & 2 & 454     & 80,000 & 80,000  & 1  & 15 & Mobile, Single Player, Shooting Gallery                \\
jc3                        & 32 & 12,695 & 80,000 & 80,000  & 1  & 20 & Console \& PC, Single Player, Open World, Action              \\
jc4                        & 7 & 1,135   & 80,000 & 80,000  & 1  & 9  & Console \& PC, Single Player, Open World, Action              \\
lis                        & 5 & 704     & 80,000 & 80,000  & 1  & 6  & Console \& PC, Single Player, Story Driven, Graphic Adventure \\
libf                       & 14 & 1,214  & 80,000 & 80,000  & 1  & 10 & Console \& PC, Single Player, Story Driven, Graphic Adventure \\ \bottomrule
\end{tabular}
\end{table}

\paragraph*{Defining the Observation Period}
Because we were interested in estimating survival time and churn probability based only on early user-game interactions it was important to define a cut-off at which point interactions were no longer be considered `early'. We call the period from the user's first interaction till this cut-off the observation period (OP). Choosing the length for the OP was not trivial as there is little indication in the literature about optimal cut-off values. Hence, we decided to visually inspect the data a-priori and extend rules proposed in \cite{drachen2016rapid, milovsevic2017early} to take into account natural inter-individual differences. Therefore, we defined the cut-off as:

\begin{equation}
\label{CutoffOP}
    \text{cutoff} = 
    \Biggl\lceil
        \dfrac
            {min(S_t, S_c)}
            {3}
    \Biggr\rceil
\end{equation}

Where $S_t$ is the total number of game play sessions and $S_c$ is the number of game play sessions before the user completed the game for the first time. In this way we take the first \sfrac{1}{3} of all played sessions for players who churned and the first \sfrac{1}{3} of played sessions before a non-churning player completed the game for the first time. We apply this cut off to the ordered list of all recorded play sessions for a specific user. We decided to use game sessions as the temporal dimension, rather than total minutes played, since we believed it better adjusted for each user's `pace' (i.e. not all the users have the possibility to play at the same frequency). Since the length of the OP has a naturally different distribution between the churning and non-churning population, we stratified our sampling technique to maintain a similar ratio of OP lengths among churners and non churners. This becomes particularly relevant for Experiment 2 and 3 where the length of the OP could leak information in the churn probability estimation task. Summarizing, if a user for example had 9 total sessions recorded, we considered the first 3 for making estimations on what happened after the 9$^{th}$. It goes without saying that at production time the OP is defined only for generating the training samples, the model can be deployed at various stages of previously unseen time series which we simulate in our experiments with the test set. 

\paragraph*{Defining the Behavioural Metrics and Targets}
We considered a set of 5 metrics, easily generalizable across games and indicative of behavioural activity, and retrieved them temporally  (i.e. over each game session during the OP), see Table \ref{metricsdescription} for a description. Additionally, we acquired a single context feature specifying the game context from where the metrics were originated. For determining the targets for our survival and churn estimation tasks, we leveraged existing literature on churn prediction \cite{drachen2016rapid, milovsevic2017early, lee2018game, perianez2016churn, runge2014churn, kim2017churn, hadiji2014predicting, xie2015predicting} and survival analysis \cite{viljanen2018playtime, demediuk2018player, lee2018game, bertens2017games}, extending existing rules to accommodate the need to define churn and survival time in single player games with a defined life cycle (i.e. non-GaaS games). We took advantage of having access to the complete session history for all users to create a churn definition which was robust to the variance in play patterns across games, as it takes into account all the recorded inter-session distances. Therefore, the criteria we adopted for defining a user as churner were both: 

\begin{enumerate}
    \item Not completing the game
    \item Being inactive for a period equal or greater to:
        \begin{equation}
            \label{inactivityrule}
            \text{inactivity} = 
            mean(\mathbf{x}) + 2.5 \cdot std(\mathbf{x})
        \end{equation}
\end{enumerate}

For better adjusting for inter-individual differences, we could have applied formula \ref{inactivityrule} to each user individually but this could have created accuracy issue for individuals with very few recorded sessions. Therefore, we opted for a conservative but more robust approach applying inactivity ($\mathbf{x}$) $\forall \mathbf{x} \in X$ where $X$ is the collection of all the considered games and $\mathbf{x}$ is the vector of inter-sessions distances in minutes for a specific game. The use of formula \ref{inactivityrule} allowed us to estimate an inactivity period which was not arbitrarily chosen but statistically defined as ‘extraordinary long’ in accordance with characteristics of play patterns in a particular game. For defining the survival time, we simply computed the total amount of Play Time in minutes for a user minus the amount of Play Time during the OP.

\begin{table}[h] \centering
\caption{\textbf{Considered Metrics over Sessions}}
\label{metricsdescription}
\resizebox{0.5\textwidth}{!}{
\begin{tabular}{@{}ll@{}}
\toprule
\textbf{Metric}            & \textbf{Description}                   \\ \midrule
{Session Time}         & Overall session duration (minutes)              \\ 
{Play Time}            & Session Time spent actively playing (minutes)    \\ 
{Delta Session}        & Temporal distance  between sessions (minutes)   \\ 
{Activity Index}       & Count of user initiated game-play-related actions. E.g.\\ 
                       & `Talk to NPC' or `Acquire Upgrade' were considered valid\\ 
                       & actions while `Click Menu' or `NPC Attacks You' were not.\\
{Activity Diversity}   & Count of unique voluntarily initiated actions \\ 
{Context}              & Name of the game taken into consideration \\ \bottomrule
\end{tabular}
}
\end{table}

\paragraph*{Data Preparation}We adopted specific data preparation procedures for each experiment. For the first analysis we collapsed the data over the temporal dimension retrieving mean and standard deviation of each considered features, to this concatenating a one-hot encoded transformation of the context metric. For the second and third experiments we kept the data in the original temporal form. In Experiment 3 only we treated the game context slightly differently, numerically encoding it and separating it from the other feature matrix. Since in Experiment 2 and 3 the length of the OP differed between users, we zero padded each sequence of considered sessions to the length of the longest sequence in the data-set. For each experiment we created a tuning and validation subsets (i.e. 20 and 80 \% of the original data-set) via stratified shuffle split \cite{scikit-learn}, employing the first for hyper-parameters searching and the second for model evaluation.

\subsection{Model Comparison}
For all experiments we applied the same procedure: first, determined the best hyper-parameters via grid search 10-fold stratified cross validation \cite{scikit-learn} on the tuning set then evaluated performance via 10-fold stratified cross validation on the validation set. In all experiments, we re-scaled the considered metric separately for each game in outliers-robust way, as in:

\begin{equation}
\label{robustscaler}
    \text{RobustRescale}=
        \dfrac
            {\mathbf{x} - Q_2(\mathbf{x})}
            {Q_3(\mathbf{x}) - Q_1(\mathbf{x})}
\end{equation}

where $\mathbf{x}$ is the feature vector to be re-scaled and $Q_n$ is the $n^{th}$ quartile for this game. The performance metric that we chose for our survival task was the Symmetric Mean Absolute Percentage Error (SMAPE), defined as:



where $N$ is the collection of all the users in the considered set and $\hat{y_i}$ and $y_i$ are respectively estimated survival time and ground truth value for user \textit{i}. SMAPE was implemented because its scale invariance allowed better comparisons of results across game contexts. For the churn estimation task the chosen metric was the F1 score (F1), defined as:

\begin{equation}
\label{f1}
    \text{F1}=
        2 \cdot 
        \dfrac
            {(precision \cdot recall)}
            {(precision + recall)}
\end{equation}

with $precision =\frac {TP}{(TP + FP)}$ and $recall = \frac {TP}{(TP + FN)}$, where \textit{TP, FP, TN, FN} stand for True Positives, False Positives, True Negatives and False Negatives. We chose the macro-averaged F1 (i.e. employing the unweighted mean of precision and recall for both classes) since our data-set was perfectly balanced.

\paragraph*{Competing Models}
As well as our novel model for joint survival time and churn probability estimation we discuss several models for disjoint estimation, learning only survival time or churn probability, in order to conduct our experiments and  compare our model with existing techniques. Furthermore, for providing a baseline comparison in our experiments we employed a mean model (MM), which generates predictions based on the average of the targets in the training set. The choice of disjoint estimation models was dictated by a series of needs: widespread usage in research and industry settings, ability to capture linear and non-linear interactions between features and most importantly capability to train on large data-sets (e.g. matrix of dimension $\approx10^6\times10^2$). Four models were employed in Experiments 1 and 2. Firstly, a variant of Regularized Regression, ElasticNet (EN) \cite{zou2005regularization}, for survival estimation and Logistic Regression (LR) for churn probability estimation. Secondly, a pair of similar Multi-Layers Perceptron Neural Networks, one tasked to perform survival time regression, MLPr, and one to perform churn classification, MLPc. We felt that given the similarities between linear models and NNs, which can be seen as a stacked version of the former with more `expressive power', the chosen algorithms constituted a natural progression in the modelling approach. For EN the best hyper-parameters were $\alpha = 0.1$ and a ratio of $0.5$ between l1 and l2 regularization. For LR an l1 regularization with $C = 0.01$. Both MLPr and MLPc employed an l2 penalty of $0.01$ and utilized a 3 layers architecture with 200, 100 and 50 hidden units. For all hidden units a $ReLU(z) = max(0, z)$ activation function was used, while an  $identity(z) = z$ and $ sigmoid(z) = \frac {1} {1 + \epsilon^{-z}}$ functions were respectively used as final activations for the MLPr and MLPc, where $z$ is a weighted sum of the hidden units of the previous layer. When training the MLP based models a small sub-set was extracted from the training set which represented 10\% of the data. This sub-set was used to evaluate convergence of the model and stop the training phase before over-fitting could occur. For both models convergence was determined if the loss did not improve for 3 epochs. The networks were trained using a batch size of 256 and optimized using the Adaptive Moment Estimation (ADAM) optimizer \cite{kingma2014adam}. Because survival time estimation is a regression task and churn prediction is classification task different loss function were used, Mean Squared Error (MSE) and Binary Cross Entropy (BCE) respectively. These are defined as:

\begin{equation}
\label{mae}
    \text{MSE}=
        \dfrac
            {1}
            {N}
            \sum\limits_{i=1}^{N}  (y_i - \hat{y_i})^2
\end{equation}
\begin{equation}
\label{bce}
    \text{BCE}=
        -\dfrac
            {1}
            {N}
        \sum\limits_{i=1}^{N}  y_i \cdot log(\hat{y_i}) + (1-y_i) \cdot log(1 - \hat{y_i})
\end{equation}

where $N$ is the size of the batch, and $\hat{y_i}$ and $y_i$ are respectively estimations provided by the model and ground truth value for the $i_{th}$ element in the batch. 

\subsection{Results}
We will first present results for each disjoint model as well as for a baseline model. Next we will illustrate in detail the performance of the BM model both in terms of it's raw accuracy as well as its capability to include uncertainty in it's output. Note that for all reported SMAPE results the smaller the better as it represents the error between the prediction and ground truth. Conversely, for F1 the larger the better since it measures how often the trained model made the correct classifications without false alarms. The probability threshold employed for discriminating between classes was set to 0.5.
\begin{table}[h]\centering
\caption{\textbf{Performance Baseline Mean Model}}
\label{baseperformance}
\resizebox{0.5\textwidth}{!}{
\begin{tabular}{@{}llrr@{}}
\toprule
\textbf{Game}  &\textbf{Model}                 & \textbf{SMAPE}      & \textbf{F1}       \\ \midrule
\textbf{hmg}   &\multirow{2}{*}{}               & $0.767 \pm 0.001$   & $0.500 \pm 0.003$ \\
\textbf{hms}   &                                & $0.581 \pm 0.001$   & $0.507 \pm 0.003$ \\
\textbf{jc3}   &\multirow{2}{*}{\textbf{MM}}    & $0.632 \pm 0.003$   & $0.499 \pm 0.004$ \\
\textbf{jc4}   &                                & $0.366 \pm 0.002$   & $0.499 \pm 0.001$ \\
\textbf{lis}   &\multirow{2}{*}{\textit{}}      & $0.404 \pm 0.001$   & $0.500 \pm 0.003$ \\
\textbf{lisbf} &                                & $0.244 \pm 0.002$   & $0.500 \pm 0.005$ \\ \bottomrule
\end{tabular}
}
\end{table}
The results from the first experiment, Table \ref{collapsedperformance}, showed how all the 4 models strongly outperformed the MM baseline, Table \ref{baseperformance}, in all games, while also achieving an overall satisfying performance. Moreover we noticed how MLPr and MLPc markedly outperformed EN and LR in both churn probability and survival time estimation across all games.  
\begin{table}[h] \centering
\caption{\textbf{Performance Collapsed Format}}
\label{collapsedperformance}
\resizebox{0.5\textwidth}{!}{
\begin{tabular}{@{}llrlr@{}}
\toprule
\textbf{Game}  &\textbf{Model}                & \textbf{SMAPE}       &\textbf{Model}                & \textbf{F1}       \\ \midrule
\textbf{hmg}   &\multirow{2}{*}{}             & $51.3 \pm 4.3$    &\multirow{2}{*}{}             & $0.591 \pm 0.004$ \\
\textbf{hms}   &                              & $33.1 \pm 2.0$    &                              & $0.624 \pm 0.004$ \\
\textbf{jc3}   &\multirow{2}{*}{\textbf{EN}}  & $42.3 \pm 0.8$    &\multirow{2}{*}{\textbf{LR}}  & $0.601 \pm 0.004$ \\
\textbf{jc4}   &                              & $35.1 \pm 0.6$    &                              & $0.663 \pm 0.002$ \\
\textbf{lis}   &\multirow{2}{*}{}             & $28.7 \pm 0.4$    &\multirow{2}{*}{}             & $0.626 \pm 0.003$ \\
\textbf{lisbf} &                              & $23.9 \pm 0.3$    &                              & $0.591 \pm 0.003$ \\ \midrule

\textbf{hmg}   &\multirow{2}{*}{}             & $30.4 \pm 0.8$    &\multirow{2}{*}{}             & $0.660 \pm 0.006$ \\
\textbf{hms}   &                              & $24.1 \pm 0.7$    &                              & $0.670 \pm 0.006$ \\
\textbf{jc3}   &\multirow{2}{*}{\textbf{MLPr}}& $36.0 \pm 0.3$    &\multirow{2}{*}{\textbf{MLPc}}& $0.654 \pm 0.004$ \\
\textbf{jc4}   &                              & $33.4 \pm 0.2$    &                              & $0.678 \pm 0.004$ \\
\textbf{lis}   &\multirow{2}{*}{\textit{}}    & $25.6 \pm 0.3$    &\multirow{2}{*}{\textit{}}    & $0.664 \pm 0.003$ \\
\textbf{lisbf} &                              & $21.9 \pm 0.2$    &                              & $0.622 \pm 0.003$ \\ \bottomrule
\end{tabular}
}
\end{table}
Following the results of Experiment 1 we tested the same modelling approaches on the unfolded version of the features, where all data points are provided rather than summary statistics. We observed a similar pattern of results, see Table \ref{unfoldedperformance}, regarding baseline and inter-models comparisons. However, it was clear that using unfolded, temporal data lead to only small improvements over the aggregated data from Experiment 1. This might be explained by the fact that the chosen modelling approaches are not explicitly designed for taking temporal structure into account, for example they have no explicit mechanics for temporal modelling such as those provided by a LSTM.
\begin{table}[h] \centering
\caption{\textbf{Performance Unfolded Format}}
\label{unfoldedperformance}
\resizebox{0.5\textwidth}{!}{
\begin{tabular}{@{}llrlr@{}}
\toprule
\textbf{Game}  &\textbf{Model}                 & \textbf{SMAPE}  &\textbf{Model}               & \textbf{F1}       \\ \midrule
\textbf{hmg}   &\multirow{2}{*}{}             & $54.5 \pm 2.4$ &\multirow{2}{*}{}             & $0.612 \pm 0.004$ \\
\textbf{hms}   &                              & $55.0 \pm 2.0$ &                              & $0.626 \pm 0.004$ \\
\textbf{jc3}   &\multirow{2}{*}{\textbf{EN}}  & $38.4 \pm 0.3$ &\multirow{2}{*}{\textbf{LR}}  & $0.607 \pm 0.003$ \\
\textbf{jc4}   &                              & $34.9 \pm 0.2$ &                              & $0.660 \pm 0.003$ \\
\textbf{lis}   &\multirow{2}{*}{}             & $30.2 \pm 0.1$ &\multirow{2}{*}{}             & $0.641 \pm 0.004$ \\
\textbf{lisbf} &                              & $23.5 \pm 0.2$ &                              & $0.578 \pm 0.003$ \\ \midrule
\textbf{hmg}   &\multirow{2}{*}{}             & $29.3 \pm 0.4$ &\multirow{2}{*}{}             & $0.683 \pm 0.005$ \\
\textbf{hms}   &                              & $22.6 \pm 0.4$ &                              & $0.682 \pm 0.004$ \\
\textbf{jc3}   &\multirow{2}{*}{\textbf{MLPr}}& $36.0 \pm 0.3$ &\multirow{2}{*}{\textbf{MLPc}}& $0.643 \pm 0.004$ \\
\textbf{jc4}   &                              & $33.1 \pm 0.2$ &                              & $0.681 \pm 0.003$ \\
\textbf{lis}   &\multirow{2}{*}{\textit{}}    & $25.6 \pm 0.2$ &\multirow{2}{*}{\textit{}}    & $0.673 \pm 0.005$ \\
\textbf{lisbf} &                              & $21.8 \pm 0.1$ &                              & $0.627 \pm 0.003$ \\ \bottomrule
\end{tabular}
}
\end{table}
Informed by the results of Experiment 1 and 2, we proceeded in evaluating the performance of our BM, Table \ref{bifurcatingperformance}, on the unfolded data. We observed how our model achieved a modest but consistent improvements in both churn probability and survival time estimation in all game contexts compared to the previous best model (MLPr and MLPc). From a visual inspection of Figure \ref{perfsurv} we can see the presence of a positive linear relationship between estimated and ground truth survival time (indicative of accordance between the two), with a roughly even distribution of error along the entire range of values. In Table \ref{confusionmatrix} we can observe how the model performance is evenly split across the two classes highlighting similar levels of precision and recall. Finally, observing the density plots in Figure \ref{fig:densurv} and \ref{fig:denchurn} we can see how the model was able to encode different levels of uncertainty through the distribution's variance of estimated values.
\begin{table}[h] \centering
\caption{\textbf{Performance Bifurcating Model}}
\label{bifurcatingperformance}
\resizebox{0.5\textwidth}{!}{
\begin{tabular}{@{}llrr@{}}
\toprule
\textbf{Game}  &\textbf{Models}               & \textbf{SMAPE}      & \textbf{F1}       \\ \midrule
\textbf{hmg}   &\multirow{2}{*}{}             & $27.5 \pm 0.1$   & $0.693 \pm 0.002$ \\
\textbf{hms}   &                              & $20.0 \pm 0.1$   & $0.701 \pm 0.003$ \\
\textbf{jc3}   &\multirow{2}{*}{\textbf{BM}}  & $34.4 \pm 0.3$   & $0.671 \pm 0.005$ \\
\textbf{jc4}   &                              & $32.5 \pm 0.2$   & $0.685 \pm 0.002$ \\
\textbf{lis}   &\multirow{2}{*}{}             & $24.6 \pm 0.2$   & $0.688 \pm 0.003$ \\
\textbf{lisbf} &                              & $20.8 \pm 0.1$   & $0.645 \pm 0.003$ \\ \bottomrule
\end{tabular}
}
\end{table}

\subsection{Model Criticism}
\lorem

\section{Dynamic Prediction of Future Behavioural Intensity}
\label{model_architecture_1}
\lorem

\subsection{Model Design}
\lorem

\subsection{Data}
\lorem

\subsection{Model Comparison}
\lorem

\subsection{Results}
\lorem

\subsection{Model Criticism}
\lorem

\section{Dynamic Prediction of Future Behavioural Intensity with Environmental and Game Covariates}
\label{model_architecture_1}
\lorem

\subsection{Data}
\lorem

\subsection{Model Comparison}
\lorem

\subsection{Results}
\lorem


\section{Discussion}
\lorem