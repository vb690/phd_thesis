When we see a professional athlete competing at an event we often find ourselves thinking "how much work they must have done to reach such level of performance". Similarly we might be surprised discovering the effort miners were putting in finding even small amount of gold nuggets during the \nth{19} century gold rush. Looking at something closer to our everyday experience, we might widen our eyes noticing how many hours we sank watching the latest TV-series or playing our favourite videogames. 

But what do all these activities have in common? They are rather different in nature but nevertheless able to elicit prolonged and vigorous behavioural responses. Indeed, it appears that human beings are capable of remarkable feats when trying to achieve goals that lead to positive and pleasurable outcomes for them. From a psychological point of view, we say that in all those instances a common set of cognitive and affective processes, which go under the umbrella of "motivation", are involved in the generation of goal-directed behaviour. 

This implies that knowing the motivational state of an individual, during a particular activity, puts us in a favourable position for understanding some aspects of their current and past behaviour as well as  predicting its intensity and likelyhood in the future. Within this general framework lies the aim of this thesis: with the current work we attempted to develop a methodology for deriving an approximate quantification of the motivational drive of individuals in situations where large volumes of observational data are present but no direct contact is possible. Indeed, without direct access to individuals, more traditional (an potentially more accurate) methodologies for inferring their motivational state are simply not feasible.

\section*{Thesis Outline}
The work carried out for this thesis, although it originated in the need to address a series of academic challenges (e.g. inferring the motivational states of individuals from large scale behavioural data) has been developed completely within an industrial setting. It is therefore the product of two separate although complementary tensions: the academic need to advance knowledge and tackle problems which have not yet found a solution and the constrains imposed by the industry to use this knowledge for developing concrete and actionable tools.

\paragraph*{Academic Outline}
\label{academic_outline}
From an academic point view we found that despite there has been noticeable interest in the estimation of the internal states generated by various psychological processes (e.g. motor control \cite{gallego2017neural}, memory \cite{derdikman2011manifold, nieh2021geometry}, visual \cite{seung2000manifold, ganmor2015thesaurus} and olfactory \cite{stopfer2003intensity} perception), less work has taken into consideration the construct of motivation \cite{mcclure2003computational, zhang2009neural}. 

In addition, most of efforts in this area have leveraged data generated in laboratory or simulation settings \cite{eyjolfsdottir2016learning, song2017reward, merel2019deep,calhoun2019unsupervised, seung2000manifold, pang2016dimensionality, luxem2020identifying, pereira2020quantifying, mccullough2021unsupervised, shi2021learning} leaving the estimation of internal states in observational settings a less explored venue. When attempts in this direction were made, the focus was mostly on animal behaviour and the adopted methodologies relied largely on completely data-driven approaches \cite{luxem2020identifying,pereira2020quantifying, mccullough2021unsupervised}. 

In this view this thesis focuses in the first place on creating a bridge connecting theoretical and computational accounts of motivation with data-driven approaches for its inference. It then leverages this theoretical framework for designing, developing and validating a methodology for approximating the motivational states of individuals using large volumes of behavioural data acquired in naturalistic settings. 

\paragraph*{Industrial Outline}
\label{industrial_outline}
One major factor limiting the inference of psychological states in naturalistic settings has been the difficulty to acquire sufficient amount of data describing the behavioural of individuals in a relatively unbiased manner. 

This issue has been partially attenuated by the, now not-so-recent, tendency to acquire this type of data through the use of telemetry systems \cite{el2016game, drachen2015behavioral}. A practice, that has seen an incredible explosion especially in industrial settings \cite{el2016game, drachen2015behavioral,EUdataregulations2018}. For this reason, a collaboration between academia and industry appears a promising venue for tackling some of the challenges outlined in paragraph \ref{academic_outline}. 

However, despite there might be an overlap between the needs of research and industrial institutions, these last ones are often laser-focused on delivering practical and actionable tools serving the ultimate purpose of improving or optimizing operations. In this view, the current thesis highlights how motivation is deeply connected with the concept of engagement, a well known construct in the videogames industry with widespread implications ranging from user experience quantification to revenue optimization. It also illustrates how the methodology introduced in section \ref{academic_outline} can be directly applied for solving tasks related to automated assessment, quantification and prediction of user engagement at large scale. This is done putting particular attention on how the methodology presented in this manuscript is able to overcome a series of limitations shared by various state-of-the-art approaches. 

The thesis will open with an overview on the literature on motivation and engagement highlighting how the latter can be seen as a behavioural derivative of the motivational state of an individual. Next, it will illustrate various approaches for characterizing and quantifying both engagement and motivation highlighting their strengths and weaknesses. The chapter closes with a review of data-driven approaches for inferring the motivational state of individuals and predicting engagement in large scale scenarios. 

Chapter two attempts to illustrate how theoretical and computational model derived from the filed of behavioural neuroscience can be leveraged for designing approaches aimed to estimate the latent states produced by motivation. It opens introducing the idea that latent states produced by motivational processes can be represented as a manifold on which observable behaviour resides. It continues illustrating a computational model of incentive salience, a particular theoretical account of reward-driven motivation, and how its underlying principles can be used for defining the architecture of an artificial neural network (ANNs). The chapter closes by proposing the idea that the ammount of motivational drive that an individual exhibits at a given point in time can be approximated by the manifold structured inferred by an ANN tasked to solve a particular type of supervised learning problem: predicting the intensity of future behavioural responses given the history of interactions between an individual and a potentially rewarding object (i.e., video-games). 

Chapter three focuses on translating the theoretical insights derived from chapter 2 into a concrete implementation. This is done through an iterative approach: starting from a simple but sub-optimal model additional components are added during three separate cycles of model specification, validation and expansion. 

Chapter four delve into the analysis of the latent representations generated by the models defined in chapter three, focusing on comparing their functional characteristics with those of attributed incentive salience. 

Chapter five shows a potential application of the work presented in the previous chapters in the context of automated engagement prediction and qualification in large scale scenarios. 

The thesis closes with a summary of the findings illustrated in each chapter, their limitation as well as recommendations and venues for future research.


