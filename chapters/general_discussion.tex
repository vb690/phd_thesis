\section{Points to cover}
\begin{itemize}
    \item Model for perfroming predictions
    \item Methodoology for extracting latent representation withing a specific framework
    \item showed how it is possible to use theory for constrains the shape of a neuroal network
    \item We are NOT modelling brain functions, we are using computational models of how the brain do specific things (i.e. attributing incentive salience) as a blueprint for designing our approach
    \item It is useful to formalize higher level constructs like engagement in terms of lower level constructs like motivation. This allows us to put in prospective what the various machine learning models aimed at predicting churn and engagement are trying to do
\end{itemize}

\section{Actual Discussion}
The present work outlined a method for embedding theory-driven knowledge in data-driven approaches, allowing to more easily interpret and test hypotheses on the representation they produce. In comparison to other works focusing on the identification of latent states (or their manifold representation) from behavioural data \cite{calhoun2019unsupervised, luxem2020identifying, pereira2020quantifying, shi2021learning, mccullough2021unsupervised}, the present methodology offers a series of advantages. It does not require the Markov assumption, it generates continuous rather than discrete state space (hence the number of hidden states doesn't need to be specified) and it relies on a more easily scalable class of algorithms. Moreover, in contrast with a general tendency of utilising completely unsupervised techniques for capturing the manifold structure underlying behavioural data \cite{calhoun2019unsupervised, luxem2020identifying, pereira2020quantifying, shi2021learning, mccullough2021unsupervised}, our methodology attempts to extract representations which obey to specific functional constrains (see section \ref{manifold_learning}) and can therefore be more easily interpreted within a specific theoretical framework. Our approach offers a convenient framework for dealing with a diverse series of tasks. It allows to produce predictions of the amount and intensity of future interactions that an individual will have with a specific object. It generates a representation that can be analyzed (similarly to what has been done in section \ref{representation_analysis}) or provided as input to other algorithms. Indeed, the encoder mentioned in section \ref{representation_analysis} can be thought of as an automatic feature extractor. This can be used to reduce complex time series data of varying length to fixed-size vectors able to describe the propensity of an individual to interact with an object. For example, the analysis presented in section \ref{partition_analysis} showed how this process could be applied for time-series partitioning of large dataset. The present work leveraged data coming from video games but the adopted approach could easily be applied to other contexts. They only key requirement is the access to behavioural quantifiers describing the amount and intensity of interactions that an individual has with a particular object, service or task. This means that natural areas of application for our approach are those relying on the remote acquisition of behavioural data (e.g. web services or online experiments) but also situations in which large volumes of experimental data are available (e.g. large multi-center studies).  

\section{Limitations and Future Directions}
The work we just presented is not exempt from limitations. First, since our approach is attempting to solve an inverse problem, the issue of uniqueness arises. Many different latent states might have produced the behavioural patterns that our model observed and there is no guarantee of a strict one-to-one mapping between the representation generated by our model and attributed incentive salience. These factors, were only partially captured by our approach as they require a higher temporal resolution (i.e. within rather than between sessions) as well as more granular indices (i.e. in-game and environmental information) than those we employed. As a consequence we can see how our approach, despite outperforming competing ones, still achieves a relatively high error rate in predicting some behavioural targets (e.g. future Absence). The behavioural profiles individuated by the partition analysis generally reflect those predicted by theories of reward-driven motivation \cite{thorndike1927law,skinner1965science,berridge2004motivation} but they also show some unexpected and potentially contradictory results (see the differences between partitions 1 and 2 and between partitions 3 and 4 in Figure \ref{partitioning}B). Given the observational setting and the unsupervised learning analysis we adopted, the explanations provided in section \ref{partition_analysis} should be taken with caution and be seen mostly as a starting point for future investigations. Clarifying the the nature of these discrepancies may require experimental work in more controlled settings. Lastly, despite the fact that our approach appeared to deal gracefully  with objects having different structural characteristics, these were limited to the domain of video games. In order to verify the generalizability of our approach, future work should include data generated from a variety of contexts (e.g. web services, online and laboratory-based experiments).