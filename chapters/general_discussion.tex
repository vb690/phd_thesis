\section{Contributions}

We will now summarise the main contributions of the work presented so far, proceeding chapter by chapter.

\subsection{Chapter 1 - Connecting Motivation and Engagement}
\label{discussion_chapter_one}
In chapter \ref{chapter_lit_review} we focused on providing an overview of the psychological process of motivation highlighting its connection with the construct of engagement.

Relying on the framework proposed by Berridge et al. \cite{berridge1998role} we illustrated how motivation can be seen as a process generating latent representations of objects informative of their capacity to produce rewarding experiences. We also illustrated how these representations act as modulatory mechanisms promoting or discouraging future interactions with said objects \cite{berridge2004motivation}.

Through a brief overview of the literature we showed how engagement with digital games can often be described in terms of the cognitive, emotional and behavioural manifestations resulting from the interactions that individuals have with particular game-object \cite{boyle2012engagement, jennett2008measuring, przybylski2010motivational}. These manifestations arise from the interaction between the internal state of the individuals and the structural characteristics of the game \cite{lucas2004sex,o2008user,jennett2008measuring,boyle2012engagement,connolly2012systematic,csikszentmihalyi2014toward}. Moreover, the surrounding environment also appears to have a role in this, acting as a modulatory factor \cite{o2008user, bialas2014cultural, vihanga2019weekly, zendle2022transnational}.

Despite attempts to connect these two processes \cite{przybylski2010motivational, nacke2011brainhex, deterding2022mastering}, a clear mechanistic description of this relationship has been elusive. One major contribution of this chapter has been to outline a unifying theoretical framework for understanding the connection between neuroscientific theories of motivation and engagement. By modifying the Engagement Process Model of O'Brein et al. \cite{o2008user}, we proposed to see motivation as a latent process arising from the interaction between individuals and (potentially rewarding) objects (i.e., videogames) and to consider engagement as its (noisy) manifestation in the observed behavioural space. 

Relying on this idea, we also highlighted the similarity between data-driven methodologies for engagement prediction in a videogame context and latent variable models used in the behavioural neuroscience literature. Despite the fact that these two approaches rely on radically different assumptions and mechanisms \cite{murphy2022probabilistic}, they both assume the existence of an unobserved component in charge of input-output mapping.

We therefore prposed that if we consider the functional goals of motivation-related processes (e.g., the predictive capacity of incentive salience attribution \cite{berridge2004motivation}) it is possible to re-frame engagement prediction models as supervised variations of the unsupervised latent variables models often used in the neuroscientific literature \cite{luxem2020identifying, mccullough2021unsupervised}.

\subsection{Chapter 2 - From Computational To Supervised Learning Models}
\label{discussion_chapter_two}
In chapter \ref{chapter_theory_modelling} we attempted to translate computational models used for describing a specific type motivation-related latent process (i.e. attributed incentive salience) into a methodology for inferring  the product of such such process from large volume of behavioural data. This was done by adapting a simulation study developed within a reinforcement learning framework into a supervised learning task.

The major contribution of this chapter has been to lay out a theoretical framework justifying the use of ANN for generating latent representations that approximate the functionalities of attributed incentive salience. In particular, based on existing computational models \cite{mcclure2003computational, zhang2009neural} we illustrated why ANNs, and their recurrent variant in particular, are well suited for this task. We also proposed a series of architectural constrains that encourage an ANN to generate latent representations compliant with the functional characteristics of attributed incentive salience. These were:

\begin{enumerate}
    \item The use of a global model architecture for obtaining a unified representation able to encompass multiple objects.
    \item The use of multi-task learning in order to force such representation to be a good holistic descriptor of observed behaviour.
    \item The use of GAM-like mechanisms in order to model the contribution of game-object characteristics (e.g., game mechanics) and environmental factors while maintaining separability and interpretability of the inferred representations.
\end{enumerate}

\subsection{Chapter 3 - Designing, Implementing and Testing the Models }
\label{discussion_chapter_three}
In chapter \ref{chapter_implementation_testing} we designed, implemented and tested the ANN architecture presented in chapter \ref{chapter_theory_modelling}.

From a theoretical point of view, the contribution of this chapter has been to validate some of the assumptions that were proposed in chapter \ref{chapter_theory_modelling}. We showed the feasibility of using a single global model for generating latent representations able to predict the intensity of future interactions between individuals and a variety of videogame "objects". We were able to confirm the important role of recurrency and non linearity for generating such representations, a finding that we replicated for all the three considered models. Finally, we showed that modelling the contribution of in-game structural characteristics and environmental factors increased the predictive power of the inferred representation, albeit only marginally. 

From an application point of view, the ANN architectures presented in this chapter (i.e., the BM, RNN and improved RNN architectures) are the first attempts (to our knowledge) to test a "global model" on large amount of behavioural data generated from multiple videogame objects \cite{perianez2016churn, liu2018semi,kristensen2019combining,liu2019micro, roohi2020predicting}. Other works investigated the potential of such "global models", but they tended to focus on a small amount of game contexts \cite{zhao2020multi} or leveraged data sources that were orders of magnitude smaller than ours \cite{liu2019micro, zhao2020multi}. Despite this, we think that the amount of data on which a learning algorithm is fitted does not tell much about its effectiveness although it certainly plays an role in the assessment of its feasibility and robustness. The methodologies illustrated in this chapter also appeared to be the first to leverage multi-task learning for more than two targets \cite{liu2019micro, zhao2020multi}in order to obtain a more holistic description of engagement from a behavioural point of view.

Finally, by using an iterative model-building approach we illustrated a methodology for dealing with complexity in a gradual but principled way. In particular, we showed how it is possible to use a pre-existing theoretical framework to inform the design and implementation of an ANN and understand how its components contribute to the improvement (or detriment) of predictive performance.

\subsection{Chapter 4 - Validating the Properties of the Representations}
\label{discussion_chapter_four}
In chapter \ref{chapter_repr_anal} we inspected and validated, from a qualitative point of view, the representations generated by our methodology.

The contribution of this chapter was two-fold. First it illustrated how the representations generated by our approach showed some of the functional characteristics associate to motivational states (e.g., the level of attributed incentive salience in particular) presented in both chapter \ref{chapter_lit_review} and chapter \ref{chapter_theory_modelling}. These were:

\begin{enumerate}
    \item The ability to distinguish between different potentially rewarding objects (i.e., videogames).
    \item The ability to differentiate between individuals based on the expected intensity of their future interactions with such objects.
    \item The ability to show the two characteristics above consistently over time.
    \item The ability to capture the dynamics underlying changes in the expected intensity of future interactions.
\end{enumerate}

Second, it provided an overview of the potential behavioural correlates associated with the representation generated by our models. This allowed us to individuate various type of behavioural "phenotypes" and to shed some light on their potential role in promoting or discouraging interactions with a rewarding object (i.e., a videogame).

From an application point of view, similarly to what we specified for chapter \ref{chapter_implementation_testing}, we showed how it is possible to leverage the representation generated by a single "global model" to derive engagement profiles from multiple game contexts. We also showed how these profiles can be derived from a diverse number of input metrics in order to take into account the behavioural characteristics of the individuals, the environment that surround them and the characteristics of the game objects with which they are interacting. To our knowledge, this is the first attempt to condense in a single coherent modelling framework both the predictive and the profiling (i.e., quantification) aspects of engagement modelling (see the overview of the literature on the topic presented in Chapter \ref{chapter_lit_review})

\subsection{Chapter 5 - Illustrating Potential Industrial Applications}
\label{discussion_chapter_five}
In chapter \ref{chapter_appliction} we reconciled the work presented in the thesis with the industrial setting in which it was developed.

The major contribution of this chapter has been to sketch the prototype of a system leveraging the modelling approach presented in this thesis for large-scale automated engagement prediction. In doing so we put particular focus on describing how the various characteristics of our methodology would fit within the different components of such system.

From an application point view, we showed how relying on a single "global model" allows the design of a coherent framework unifying all the components of the system, from data parsing and pre-processing to model generation and serving. Such "global model" can serve multiple game contexts (or multiple game providers) at the same time allowing to pool information across them. This gives access to a higher degree of effectiveness and generalizability while potentially decreasing maintenance and deployment costs. Finally we showed that relying on a methodology that focuses not only on prediction but also on representation learning allows the design of systems able to cover multiple functions: going from predictive analytics, to feature extraction and sharing.

\section{Conclusion}
\label{conclusions}
We have outlined a methodology for approximating motivation-related latent states in situations where large volumes of behavioural data are available but no direct contact with the individuals that generated them is possible. Importantly, our approach attempts to respect both computational and theoretical constraints provided by previous works in the field of behavioural and affective neuroscience. 

In doing this, we showed that it is possible to embed theory-driven knowledge in data-driven approaches, allowing us to more easily interpret and test hypotheses regarding individuals' latent states. In comparison to other works focusing on the identification of latent states (or their manifold representation) from behavioural data \cite{calhoun2019unsupervised, luxem2020identifying, pereira2020quantifying, shi2021learning, mccullough2021unsupervised}, our methodology offered a series of advantages: It did not require the Markov assumption, it generated a continuous rather than discrete state space and it relied on a more easily scalable class of algorithms. 

In contrast with a general tendency of using completely unsupervised techniques for capturing the manifold structure underlying behavioural data \cite{calhoun2019unsupervised, luxem2020identifying, pereira2020quantifying, shi2021learning, mccullough2021unsupervised}, we proposed a methodology based on supervised learning in order to obtain representations compliant with specific functional constraints. This allowed us to better frame the current work within existing neuroscientific theories of human motivation. 

Despite the fact that our work never attempted to model the mechanics of psychological processes directly, we showed how it is possible to leverage computation model of cognitive and affective functions to design data-driven methodologies for inferring the product of such processes. This is important in two fundamental ways. First, it allowed us to construct data-driven solutions in a more principled way, rooting their functional form in the nature of the problem they were trying to solve. Second, it made it easier to specify theory-informed hypotheses \textit{a-priori} and compare them against the behaviour of an otherwise black-box approach.

By connecting this methodology to the construct of engagement we showed how it could be leveraged within an industrial setting as the machine-learned component of a system designed to perfrom cross-games automated engagement prediction and quantification.

The present work leveraged data coming from video games but the approach could easily be applied to other contexts. They only key requirement would be the access to behavioural quantifiers describing the amount and intensity of interactions that an individual has with a particular object, service or task. This means that natural areas of application for our approach are those relying on the remote acquisition of behavioural data (e.g. web services or online experiments) but also situations in which large volumes of experimental data are available (e.g. large multi-center studies).

\section{Limitations and Future Directions}
\label{discussion_limitations}
Of course, this work is not exempt from limitations:

Because we are solving an under-determined inverse problem, the issue of uniqueness arises. Many different latent states might have produced the behavioural patterns that our model observed and there is no guarantee of a strict one-to-one mapping between the representation generated by our model and attributed incentive salience.  More effort in future research must focus on obtaining a clear formulation of the computations carried out by our architectures. Alternatively, validation could be carried out by comparing the representations generated by our architectures with those inferred from data gathered through laboratory or simulation experiments. An interesting venue for future research would be to evaluate similarities and differences between representations generated by learning algorithms designed according to different models of human motivation (e.g., error-based \cite{schultz1997neural} and error-free \cite{friston2012active} models).

Although the findings illustrated in chapter \ref{chapter_repr_anal} suggest a similarity between the functional characteristics of the representation inferred by our approach and the construct of attributed incentive salience these are the result of mostly qualitative, descriptive or exploratory analyses. Future work could aim to derive quantifiable indices from the generated representations (in a similar way to the work carried out by Zhang and colleagues \cite{zhang2009neural}) or to assess the degree of overlap between model-generated and human-generated representations (as in the work conducted by Roads and Love \cite{roads2021enriching}).

Despite we proposed a way to constrain an ANN architecture to obtain more interpretable latent representations, our approach is still far from being compliant with Occam's razor. Given a growing literature on the computational substrates of motivation-related cognitive processes \cite{schultz1997neural, mcclure2003computational, zhang2009neural, wang2018prefrontal} an interesting venue for future research would be to move from an approach based on ANNs to one based on latent variable models (e.g., state-space models or methodologies similar to those proposed by Calhoun et. al \cite{calhoun2019unsupervised})

This type of approaches, although less scalable, offer a higher degree of control on the computations underlying the dynamics of the latent representations. They allow to specify complex but parsimonious (in terms of number of free parameters) models with very well defined structural forms (e.g., they require the explicit specification of state equations). Utilising an iterative model building approach similar to the one presented in this thesis it would then be possible to more accurately assess the role of various components of the model while also obtaining more precise functional forms with parameters that are directly interpretable.

The mapping from the inferred latent representation to the observed behavioural space  presented in section \ref{partition_analysese}, despite producing results broadly in line with theories of reward-driven motivation \cite{thorndike1927law,skinner1965science,berridge2004motivation} appeared to show some unexpected and potentially contradictory results. Given the observational setting of our work and the unsupervised approach we adopted, the explanations provided in section \ref{partition_analysese} should be taken with caution and be seen as a method for generating testable hypotheses used as the starting point for future investigations. Clarifying the the nature of some of the observed discrepancies may require experimental work in more controlled settings. 

The limited computing resources available during the development of the thesis didn't allow a more extensive evaluation of our methodology. For example, despite the availability of a large pool of videogame titles we had to select a relatively small number of them in order to be able to effectively run our experiments. A similar issue arose in connection with the number of considered individuals, despite we were able to conduct experiments on sample sizes that were (at the time of conducting the experiments) unprecedented in the videogame literature these were only a relatively small fraction of the user base we had access to. Finally, more generous computing resources would have allowed for a higher number of experimental replications, better hyper-parameter tuning and more detailed exploration of architectural variations. In this view, future research should focus on scaling and testing our approach against considerably higher volumes of data.

Despite we proposed the design of a system leveraging our methodology for automated engagement prediction, this was never tested within a production environment. Future effort should focus on evaluating the feasibility of deploying our solution within an industrial setting and most importantly thoroughly documenting eventual advantages and pitfalls that might emerge.

Lastly, despite our approach having appeared to deal gracefully with objects with different structural characteristics, these were limited to the domain of video games. In order to verify the generalizability of our approach, future work should include data generated from a variety of contexts (e.g. web services, online and laboratory-based experiments).  

