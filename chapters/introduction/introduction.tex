\section{General Introduction}
\lorem

\section{Thesis Outline}
The goal of this thesis is to design, develop and test a methodology for approximating the motivational state of individuals in situations where large volumes of observational data are present but it is not possible to have direct contact with the individuals that generated them. Chapter one with a general outline on motivation and reward-driven motivation in particular. Stemming from here it highlights, how from a behavioural point of view, engagement can be seen as a derivative of the motivational state in which an individual is. Next, It continues showing how motivation can be conventionally assessed highlighting strengths and weaknesses of different approaches with particular attention to applications in large scale scenarios. The chapter closes on a review of works for estimating motivation and engagement at large scale with a particular focus on applications in a video game context. Chapter two aims to illustrate how we can leverage knowledge from theories of motivation for designing data driven solution for the estimation of the motivational state of individuals and engagement prediction. It opens with the illustration of a computational model of incentive salience, a particular theoretical account of reward-driven motivation, and it continues showing how the principles of this model can be used for defining the architecture of an artificial neural network. The chapter closes with a series of experiments aiming to assses the profiiency of the derived model through model comparison. Chapter three aims to illustrate how it is possible to extract and inspect the representantion generated by an articial neural network. It opens illustrating the concept of representation learning in the context of artificial neural network and it proceeds illustrating a series of methodologies for analyzing this representations and deriving insights from it. Chapter four illustartes how the model designed in chapter 2 can be extended and improved. Chapter 5 shows a potential application of the work carried out in this thesis for large scale engagement estimation in a videogame context.


