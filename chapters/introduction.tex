When we see a professional athlete competing at an event we often find ourself thinking "how much work they must have done to reach such level of performance". Similarly we might be surprised discovering the effort miners were putting in finding even small ammount of gold nuggets during the \nth{19} century gold rush, and we widen our eyes looking at how many hours we sank watching the latest Netflix series or playing our favourite videogames. But what do all these activities have in common? They are very different from each other but nevertheless able to elicit prolonged and vigorous behavioural responses. Indeed, it appears that human being are capable of remarkable feats when trying to achieve goals that lead to positive and pleasurable outcomes for them. From psychological point of view, we say that in all those instances a common set of cognitive and affective processes, which go under the umbrella of "motivation", are involved in the generation of goal-directed behaviour. This implies that knowing the motivational state of an individual, during a particular activity, puts us in a favourable position for interpreting their behaviour and predicting its future intensity. Within this general framework lies the aim of this thesis. With the current work we attempted to develop a methodology for approximating the motivational states of individuals in situations where large volumes of observational data are present but a direct contact with the individuals is not possible. Being this thesis developed within an industrial setting it posses two souls one more theoretical and academic oriented and one more applicative and industry oriented.

\section*{Thesis Outline}
The work carried out in this thesis is the product of two separate although complementary tensions: the academic need to advance knowledge and tackle problems which has not yet found a solution and the industry need to 


\paragraph*{Academic Outline}
From one side it aims at bridging the gap between theoretical account of motivation and data driven approaches for its inference. With the aim to design, develop and validate a methodology for approximating the motivational states of individuals from solely behavioural data. 
\paragraph*{Industrial Outline}
On the other side, the same methodology is designed for application in industrial and experimental settings for analyzing and predicting behavioural at large scale. In particular this thesis focus on the application of our methodology to the context of video-game engagement prediction and characterization.
\\
\\
The Thesis will open with an overview on the literature on motivation, engagement highlights how, from a behavioural point of view, engagement can be seen as a derivative of the motivational state in which an individual is. Next, It continues analyzing how engagement and motivation are conventionally assessed highlighting strengths and weaknesses of different approaches with particular attention to applications in large scale scenarios. The chapter closes on a review of works for estimating motivation and engagement at large scale with a particular focus on applications in a video game context. Chapter two aims to illustrate how we can leverage knowledge from theories of motivation for designing data driven solution for the estimation of the motivational state of individuals and engagement prediction. It opens with the illustration of a computational model of incentive salience, a particular theoretical account of reward-driven motivation, and it continues showing how the principles of this model can be used for defining the architecture of an artificial neural network. The chapter closes with a series of experiments aiming to assess the proficiency of the derived model through model comparison. Chapter three aims to illustrate how it is possible to extract and inspect the representation generated by an artificial neural network. It opens illustrating the concept of representation learning in the context of artificial neural network and it proceeds illustrating a series of methodologies for analyzing this representations and deriving insights from it. Chapter four illustrates how the model designed in chapter 2 can be extended and improved. Chapter 5 shows a potential application of the work carried out in this thesis for large scale engagement estimation in a video game context.


