\section{Introduction}
\label{motivation_engagement_introduction}
\lorem

\section{Motivation as a Reward-Driven Process}
\label{motivation}

The construct of motivation is a key concept for understanding why individuals seek out specific objects or experiences at particular times and why they react in particular ways when encountering stimuli considered of particular relevance [1]. In this view motivation can be defined as the process leading the modulation and reiteration of goal directed behaviours that once reached exerts rewarding (i.e. pleasurable) effects on the individual [2]. What is a common confound when defining this phenomenon is to partition it in different sub-components each one supposedly representing a different version of the motivational process (e.g. the motivation of X, the X motivation) [3]. The motivational process remains always the same, what changes is the nature of the goal producing the motivated behaviour [2]. If we represent motivation as a vector, its length would be the intensity of pursuit (or the amount of goal directed behaviour) while the angle the focus on a specific goal [2]. Summarizing we can say that the behaviour of an individual is motivated by the expectancy of pleasurable outcomes derived by the goal the behaviour is aiming to reach [1]. The aforementioned statement becomes particularly relevant for those spontaneous activities which are not driven by the fulfilment of fundamental needs (e.g. hunger or thirst) or by the avoidance of negative consequences (e.g. cognitive or physical pain), indeed for these activities what matters is the nature of the goal which may largely vary among individuals. Having a clear definition of the motivational process in place will help us understanding how some of the findings regarding motivational factors in videogames can be reframed in a behavioural context.

\subsection{An Historical View on Reward-driven Motivation}
\label{motivation_hist}
One of the early theorizations of human motivation proposed that individuals were motivated by the expectance of incentive outcomes. These expectations are formed through a process in which an individual learn the existence of an association between the actions he performs and the potential pleasurable outcomes associated to them [1]. Extending on this conceptualization, it has been proposed that the learning process does not only provide a way for forming the expectation of a pleasurable outcome in response to a specific behaviour but also allow to perceive the behaviour itself as source of hedonic reward [1], this concept of hedonic reward will be clarified later in this section as it constitutes an interesting point of connection with existing models of motivational factors in videogames. A third theoretical formulation, which fuses the two aforementioned approaches, making use of  concept related to learning through reinforcement formulated that a stimulus and the behaviours associated to it become relevant and salient for an individual as a consequence of learning its rewarding properties [1]. What written in this paragraph seems very distant from our initial starting point, to a certain extent an over-simplistic lab rat view on human behaviour, but what has been described here is one of the fundamental and most elegant (in terms of complexity to explanatory power ratio) mechanism able to explain and describe at why people do what they do, why they keep doing it and why they do that for a specific amount of time. Obviously taking this as the ultimate explanatory approach would be not just naïve but wrong as many layers of complexity need to be applied for approaching a good description of the motivational process in humans, however we can see some similarities with our previous overview on engagement: the relationships between the individual and the activity preformed, the ability of the activity in providing pleasurable experience and the necessity of feedback (i.e. reward) for maintaining engagement.

\subsection{Learning mechanisms: classical and operant conditioning}
\label{classical_operant_cond}
We will now illustrate the concepts of classical and operant conditioning, basic learning processes which are fundamental for introducing the higher level concept of incentive salience later on. For understanding the processes of classical and operant conditioning it is worth defining what a reinforcing object and process are. We can describe as reinforces those object having the capability to alter the probability of appearance of a specific behaviour [34, 35, 36].  On the other hand the reinforcing process identify the development of a condition in which a specific behaviour becomes more probable when followed by particular reinforcing events and becomes less probable when this last ones are not present anymore [34]. Reframing this concepts in a videogame context we can hypothesize that in-game contents or mechanics act as reinforcing objects when they produce positive outcomes for the player, in a cyclic process in which interacting with the game might produce positive reactions player-activity interactions which in response will make new interactions with the game more probable. 
We will now see the how classical and operant conditioning operationalize the aforementioned concepts of reinforcing object and process.
CLASSICAL CONDITIONING
Classical conditioning describes a learning process in which, independently from the activity of an individual, the repeated pairing of two stimuli will cause one to acquire the eliciting properties of the other [36]. Employing the aforementioned concepts of reinforce, the repeated pairing of a neutral object with reinforcing consequences will imbue the first with reinforcement properties. This concept is extremely simplistic and not suitable for describing human behaviour, however succinctly illustrate a very basic and common learning process other than being the ground from which operant conditioning stems.
OPERANT CONDITIONING
Operant conditioning extends on the concept of classical condition introducing the agency of the individual. The concept refers to the process in which the frequency of a behaviour tends to increase when precise consequences are associated to it [35]. In this view, an operant is formalized as a goal directed behaviour while all the elements reinforcing the re-iteration of this behaviour are called reinforces [35]. The learning process here results from the relationship between a behaviour and its consequences, therefore the probability of behaviour to take place is related to its capability to produce reinforcement [34]. Again, reframing this concept in a videogame context the goal directed behaviour would be the interaction with the game environment while the reinforces are the consequences generate by the interaction itself, which might come in form of internal state only (i.e. pleasurable experiences) or also in the form of in game rewards (i.e. game feedbacks). 

\subsection{Reward and Incentive Salience Attribution}
\label{incentive_salience}
Until now we have mainly used the terms reinforces and incentives for identifying stimuli able to drive and shape behaviour, but when it comes to define effective reinforces, is not just a matter of merely pairing a behaviour with a stimulus but the stimulus itself has to have particular properties. In this view, stimuli able to generate pleasurable feelings in the individual are the best candidates for being effective reinforces,  they are said having ‘rewarding properties’.  But what is, and how can be defined the reward?  The reward is a process generated in response to a stimulus making it desirable for its capacity to generate pleasurable responses. In this view, for being able to generate rewarding response, a stimulus needs two fundamental properties: it has to be wanted (i.e. it acquires the capacity to become desirable) and liked (i.e. it has to be able to generate pleasure in the individual) [37]. But how a particular object acquires this properties? Here comes in action the aforementioned learning processes: the repeated pairing between a stimulus and the reaction it generates in the individual imbue the first with rewarding properties. Moreover, through the aforementioned learning process (i.e. operant conditioning) not just the stimulus itself but also the connected instrumental behaviour will likely acquire the same rewarding properties [37]. A useful distinction that can be me here is between stimuli having primary and secondary reward properties:
Primary rewards 
These are type of stimuli having intrinsic rewarding properties due to their linkage with essential evolutionary needs (i.e. satisfaction of homeostatic needs),  the rewarding properties don’t have to be learned but already present, up to a certain extent, in the stimulus itself [45]. 
Secondary rewards
these are type of stimuli which don’t hold an innate capacity to generate rewarding experiences, their capacity to generate pleasurable experiences and consequently being wanted is strongly related to the occurrence of some of the aforementioned learning processes [45].

Despite this distinction might seems superfluous at a first glance, it has a particular relevance for understating the framework in which videogames lies. No element in a videogame context possesses intrinsic rewarding properties, here the learning process is essential for allowing the element to acquire the aforementioned rewarding properties. The attractive property of an in-game element is not present before the user start playing the game but it might be learned through the user – game interactions occurring during the playing activity.

The approaches proposed by Bolles, Bindra and Toates,  provide an account of reward-based motivation but they assume that there is no distinction between the affective dimension of an incentive (i.e. how pleasurable it is) and the purely motivational aspect of it (i.e. how much goal directed behaviour it can produce) \cite{bindra1978adaptive,toates1994comparing}. Expanding on this, Berridge and Robinson proposed that the motivational process controlling the interaction between individuals and objects might not be a unitary mechanism but rather a composite process having specific and dissociable components which rely on specialized neurobiological mechanisms, namely: \emph{liking}, \emph{wanting} and \emph{learning} \cite{berridge1998role,berridge2009dissecting,smith2011disentangling}.

\paragraph{Liking}
\label{liking}
The \emph{liking} component describes the pleasure expected by an individual when interacting with an object \cite{berridge2009dissecting}. It is responsible for the hedonic quality of an experience and acts as a signal indicating that interacting again with that object might be beneficial. Despite the fact that \emph{liking} plays an important role in the incentive salience hypothesis of motivation it is difficult to measure it outside controlled laboratory environments \cite{berridge1998role} and it will not form a central theme of this research. Instead, we will focus on the "wanting" and "learning" components.


\paragraph{Wanting}
\label{wanting}
The \emph{wanting} component, or "incentive salience", has the function of generating and holding latent representation of objects and behavioural acts and of attributing value to them through learning mechanisms. These "valued representations" can then be used by action selection systems in order to make certain behaviours more likely \cite{ikemoto1996dissociations,berridge1998role,mcclure2003computational,berridge2004motivation}. As a consequence of this, when an object is attributed with incentive salience it will more likely draw the subject's attention and become the focus of goal directed behaviours \cite{berridge2004motivation}. Interestingly, \emph{wanting} seems to be more than a simple form of value-caching but rather a dynamic process in constant change \cite{robinson1993neural,zhang2009neural,tindell2009dynamic,berridge2012prediction}. This is because the saliency of an object depends both on its attributed value but also on the state of the individual interacting with it. A change in the individual's internal state can dampen, magnify or even revert the amount of attributed salience. \cite{robinson1993neural,zhang2009neural,tindell2009dynamic,berridge2012prediction}. It is important to note that \emph{wanting} is not the hedonic expectation associated to an object, (which is designated by \emph{liking}), but rather the process promoting the approach towards an object and the interaction with it \cite{berridge2009dissecting,robinson2015roles}. Despite the fact that \emph{liking} and \emph{wanting} are often correlated (i.e. I want what I like and vice versa) they can occasionally be triggered separately: addictive behaviours for instance are a notable example of \emph{wanting} without \emph{liking} \cite{robinson1993neural}. The functional dissociation between these two components is linked to differences in the underlying neurobiological substrate \cite{berridge2009dissecting,smith2011disentangling}. Neurotransmitters and brain areas responsible for \emph{wanting} appear to be more numerous, diverse and easily activated than those for \emph{liking} \cite{berridge2009dissecting,robinson2015roles}. As a consequence, increased incentive salience can be obtained by raising dopamine levels in many portion of the striatum without the need for the synchronized activity in other areas \cite{berridge2009dissecting,smith2011disentangling,meyer2015motivational}. This implies that the \emph{wanting} component tends to produce more robust behavioural indicators in the form of increased amount and frequency of interactions between an individual and an object \cite{berridge1998role}, which makes it a promising candidate for behavioural studies in conditions where strict experimental control is not possible.


\paragraph{Learning}
\label{learning}
Classical conditioning describes a learning process in which, independently from the activity of an individual, the repeated pairing of two stimuli will cause one to acquire the eliciting properties of the other [36]. Employing the aforementioned concepts of reinforce, the repeated pairing of a neutral object with reinforcing consequences will imbue the first with reinforcement properties. This concept is extremely simplistic and not suitable for describing human behaviour, however succinctly illustrate a very basic and common learning process other than being the ground from which operant conditioning stems.

Operant conditioning extends on the concept of classical condition introducing the agency of the individual. The concept refers to the process in which the frequency of a behaviour tends to increase when precise consequences are associated to it [35]. In this view, an operant is formalized as a goal directed behaviour while all the elements reinforcing the re-iteration of this behaviour are called reinforces [35]. The learning process here results from the relationship between a behaviour and its consequences, therefore the probability of behaviour to take place is related to its capability to produce reinforcement [34]. Again, reframing this concept in a videogame context the goal directed behaviour would be the interaction with the game environment while the reinforces are the consequences generate by the interaction itself, which might come in form of internal state only (i.e. pleasurable experiences) or also in the form of in game rewards (i.e. game feedbacks).

We will now illustrate the concepts of classical and operant conditioning, basic learning processes which are fundamental for introducing the higher level concept of incentive salience later on. For understanding the processes of classical and operant conditioning it is worth defining what a reinforcing object and process are. We can describe as reinforces those object having the capability to alter the probability of appearance of a specific behaviour [34, 35, 36].  On the other hand the reinforcing process identify the development of a condition in which a specific behaviour becomes more probable when followed by particular reinforcing events and becomes less probable when this last ones are not present anymore [34]. Reframing this concepts in a videogame context we can hypothesize that in-game contents or mechanics act as reinforcing objects when they produce positive outcomes for the player, in a cyclic process in which interacting with the game might produce positive reactions player-activity interactions which in response will make new interactions with the game more probable. 
We will now see the how classical and operant conditioning operationalize the aforementioned concepts of reinforcing object and process.

\section{A Behavioural Perspective on Engagement}
\label{engagement}
Playing games has always been present in human history as an occupation aiming to entertain and relax [42], it can be defined as a free-time activity with spatial and temporal boundaries able to intensely absorb who is involved in it [42]. A special case of the broader group of games are those being delivered and experienced in a digital format (i.e. videogames) which since various decades has been substituting more traditional ludic activities [42, 43]. This phenomenon has been well reflected both in terms of number of people involved in playing videogames as well as in the amount of time spent engaging in this activity [44]. One of the main reasons for this explosive phenomenon relies on the fact that videogames seems to be perfect medium for delivering enjoyable experiences [32], consequently holding a strong potential to engage and retain users involved in the playing activity. Various attempt has been made to understand engagement in videogames, both at the level of process and factors driving and influencing it [32].

\subsection{Factors Driving Engagement}
\label{factors_engagement}
The literature reports various theoretical approaches  for addressing the possible different mechanism producing engagement [32], we will briefly illustrates some of the most prominent ones focusing in a second moment on a specific framework connecting engagement to the motivational factors provided by videogames. Focusing on the motivational factors driving engagement is a well suited ground for building a methodology aiming to assess and evaluate engagement via behavioural measures, this because we can take advantage of existing theories describing human behaviour via the interaction between the individual (i.e. the player) and the stimuli (i.e. the game features) present in the environment (i.e. the game world).

Flow
A classical construct employed in the videogame literature for explaining the phenomenon of engagement, is the concept of flow developed by [33]. The formulation of this construct prescribe that when an individual is absorbed in an activity perceived as valuable they will experience a rewarding state of optimal pleasure constituting the fuel for of  engagement process [32]. In this view the conditio sine qua non for the flow state to arise is a balanced combination of the individual’ skill level and the difficulty of the task in which they are involved. Despite offering an interesting point of view, the concept of flow as a framework for explaining engagement in digital games might be prone to the fallacy of circular reasoning: is a user engaged in a specific activity because this provides the optimal flow experience or this last one is a bi-product of being engaged in the activity itself? 

Immersion
A construct linked to the concept of flow is immersion, which differently from flow is concerned with the specific, psychological experience of engaging with a computer game [46]. Immersion is referred as the experience of engaging in one moment in time with a videogame rather than being a factor driving engagement itself [46]. The experience of immersion involves loosing track of time, space and having a sense of being in the task environment, all of this occurring as a result of a good gaming experience. What emerges from the brief overview of the concept of immersion is that rather than posing itself as a factor influencing or driving engagement it seems to provide an alternative approach for describing and characterizing it.

Uses and gratifications
Uses and gratification theory states that users possess different motives characterizing their gratification-seeking behaviours and that these can be satisfied through media consumption [47]. Engagement in this type of activities (i.e. media consumption) is justified by their ability to meet the motives driving the user behaviour. Another point that this theoretical approach assumes is that media audience are not passive but rather variably active, concept that is particularly relevant when evaluating the interaction between players and videogames, since the first have an active role in determining how the latter changes over time[47]. Despite its relatively vague formulation, uses and gratification theory introduce two important concepts: first that users engage in a spontaneous activity (i.e. media interaction) in search of some form of gratification and second that this interaction is not passive but an active process in which the user interact with the medium.



\subsection{The Engagement Process Model}
\label{eng_proc_model}
Various attempts have been made for formalizing the concept of ‘engagement in digital games’, but this appears to be a non-trivial task with the resulting theoretical framework being heterogeneous and not unified [32]. We will employ the work of [40] for providing a general overview on engagement. In [40] engagement is defined as:
‘…a quality of user experiences with technology that is characterized by challenge, aesthetic and sensory appeal, feedback, novelty, interactivity, perceived control and time, awareness, motivation, interest and affect’.
Building on this, [40] avoids to provide an exact definition of engagement but rather describes it as a process with distinct phases each one possessing peculiar attributes.
Point of engagement
It Is the starting point of the engagement process, it configures as the moment in which the user’s attention is directed towards a specific activity due to properties of the activity itself or for its capacity to fulfil specific motivational drives that the user might have.
Period of engagement
It is the period during which the user is having a sustained involvement into the activity and it is fostered by the ability of the activity to provide appropriated feedbacks, novel information and features.
Disengagement
It is defined as the moment in which the users makes an internal decision to stop the activity or when external factors force them to not engage in it anymore. The internal factors are usually connected to loss of interest or pressure derived by the time passing, external factors instead relate to the inability of the activity to provide novelty or to the occurrence of distractors during the activity.
Reengagement
It identifies the moment in which the user returns to the activity after disengagement occurred. This can happen both at short term and long term and it is the result of positive past experiences with the activity, which are usually linked to be exposed to rewarding incentives or novel content within the activity.
What emerges from the description of engagement as process seems to be that a mix of properties of the activity, internal states of the user and ‘environmental’ factors external to the activity itself controls the engagement’s quality and quantity. This consideration will be functional for the theoretical framework we are trying to build through this review.

SUMMARY
The aim of this brief review was not to define or once-for-all clarifying the concept of engagement in digital games (it would be almost impossible other than out of scope) but rather to identify a possible fil rouge connecting the existing works. What emerged seems to point in the direction of engagement being a second order factor emerging from (or generated by) a series of subjective experiences linked to the interaction between the user and the playing activity. What seems to connect all these different approaches seems to be the description of a process in which the interaction between the user and the game activity generates feelings of enjoyment and pleasure ultimately contributing to the engagement in the activity itself [32]. This interpretation will become particularly useful when we will address the concept of reward and incentive salience later on.

\section{Engagement as a Derivative of Motivational Processes}
Motivation and engagement seems to be interchangeable terms but we argue that they are qualitatively different. Motivation pertain the state of an individual with resect to achieving a particular goal, perfroming a particular action or interacting with a particular object. Engagement on the other hand describes the quantity and quality of the interaction both from a behavioural and experiential point of view. From now on the relationship between motivation and engagement will be assumed we will have to rember that engagement, from a behavioural point of view, is a derivative product of the motivational state in which an individual is. 


We previously mentioned the concept free time activity which is not driven by external factors, in this sense playing videogames is a fully-fledge part of this type of activities: no external incentives justify engaging in it, what instead motivates the playing activity are the inherent properties of the experience [4]. Reframing the presented definition of motivation in a videogame context we can define the amount of interaction with the game as the motivated behaviour and the in-game elements or features as the inherent properties (or goals) the player is playing for, namely the incentives which justify the gaming behaviour [4]. This perspective, although functional, may be too simplistic: human behaviour (and gaming is not an exception) emerges from the interaction between an individual and the surrounding environment, where in a videogame context the environment is defined by all the elements constituting the game while the individual is identified as the collection of personal characteristic defining the player. This relationship is of pivotal importance when designing a game: the ability to construct an environment able to produce positive (i.e. pleasurable) emotional manifestations in the players constitutes one of the major tools for having them engaged in the gaming activity [5]. Various attempt to individuate these characteristics in players based on their in-game behaviour and propensity towards specific game feature can be individuated in the literature, we will now briefly illustrate some of the major model which tried to formalize these differences.
THE EARLY DAYS
In his seminal work [38] tried to identify different approaches that players might have had in playing Multi User Dungeons (MUDs), an early version of the modern Massively Multiplayer Online Role Playing Games (MMORPGs). Projecting the players’ attitudes toward the game on two axis: action versus interaction and world oriented versus player oriented, [38] proposed 4 different, mutually exclusive, categories each one with peculiar attributes:
Achievers (action oriented towards the game world)
These were players interested in mastering the game, aiming to build a status within the game towards their interaction with the environment.
Explorers (interaction with the game world) 		
These were identified as players aiming to be surprised by the game world, seeking the stimulation derived by the discovery of new areas in the game world, attributing value to the acquisition of knowledge related to the game
Socializers (interaction with other players)
These were players driven by the perspective of interacting with other players in the game world, deriving satisfaction from their friendship, contacts and social influence within the game.
Killers (action oriented towards other players)
They are interested in demonstrating their superiority over other players posing great value  in the reputation obtained through  in-game fighting skills.
Despite this early formulation only consider a specific subset of games with their inherent characteristics and lacks in any kind of empirical validation the work of [38] is the starting point from which most of the later models of player characterization stems. Most importantly this is one of the first attempts introducing the concept that the in game behaviour might be indicative of specific propensities towards activities that the player finds satisfactory.
AN EMPIRICAL EXTENSION
As specified before, the taxonomy proposed by [38] was built on assumptions that were never tested and with the proposed categories possibly having a certain degree of inter-correlation it was necessaire to find a more robust way for assessing the players’ motivational drives [19]. In a work by [19] the author aimed to fill the aforementioned gaps, developing a questionnaire for evaluating the various motivational factors driving the involvement in video game playing, more specifically MMORPGs. After gathering information from a large sample of players and performing a first round of dimensionality reduction, 10 major factors emerged that were then condensed in subsequent 3 additional ones via ulterior dimensionality reduction performed over the previously obtained components. The three motivational factors emerged by the work of [19] were:
Achievement
This motivational factor is derived by advancing in the game, exploiting its mechanics or competing with it.
Social
This factor is represented through in game mechanics centred on socializing, creating relationship and developing teamwork.
Immersion
This factor is represented by the discovery, role-playing and customization mechanics of a game
The model proposed by [19] introduced two interesting variations on what has been done by [38]: the components are thought to be not necessarily mutually exclusive and the focus is shifted from a characterization of the players to a characterization of the in-game elements with which the players interact. Focusing more on the in game elements rather than on the player offers the advantage to reduce the complexity of the entity the model is trying to describe. However the focus was still on a specific game genre and the use of self-report measures for deriving the factors might pose some constrains on the linkage with actual in-game behaviour.
ADDITIONAL FACETS
For overcoming the limitation of a taxonomy heavily influenced by a specific game genre, [39] developed a more extensive system with the intended to capture the players’ preferences over particular game mechanics regardless of the specific game played. The categories proposed by this model are:
Seekers 
These are players driven by in game mechanics which produces interest and satisfy curiosity, for instance discovering and exploring the game world.
Survivors
These are players driven by thrilling experiences provided by the game and therefore by the ability of the game environment to generate arousal.
Daredevils 
These are players motivated by the thrill derived from taking risks in the game.
Masterminds 
These are players who enjoys solving puzzles and finding the best strategies to adopt within the game.
Conquerors
These are player players who derive satisfaction not just from overcoming the challenges provided by the game from the struggles characterizing the process.
Socializers 
These players find the motivational drive for playing the game in the interaction with other players: talking with them, helping them or simply staying with them in the game world. 
Achievers 
These are players driven by reaching long term goals, often aiming to fully complete the game.
In the model proposed by [39] the focus is moved again from the in-game elements to the player, proposing this time a higher level of specificity in the definition of the typologies. Despite the work by [39] might hold higher descriptive power due to the introduction of new facets, we want to point out that this is a process that might grow proportionally with the number of game features taken in consideration, potentially obtaining as many typologies as the number of mechanic in each possible game. We voluntarily didn’t spent words on the proposed link between the model and ‘neurobiological’ and personality components, the reason for this is that the presented link to neurobiology is purely anecdotal, not supported by any evidence and adopt  a conceptualization of the human brain that is, at the very least, primitive. Evaluating the relationship between the facets and personality traits could have been an interesting angle to explore, but the use of the infamous Mayers-Briggs model and the adoption of a rather fragile methodology make the presented results totally intractable.
A MODEL INFORMED BY PSYCHOLOGY
What has been illustrated so far show how most the model aiming to address motivational factors in videogames arises directly for the videogame field itself. Using a different approach [4] employed a psychological model for explaining in an empirical way the process through which videogames drive sustained engagement. The underlying idea was that videogames, lacking in the presence of external incentives, provide appeal via the inherent properties of the playing experience. In the original formulation by [3] humans are driven in their every-day life by the satisfaction of basic needs, these results as more effective motivational factors than any kind of  external incentives. We will now present how the formulation by [3] has been employed by [4] for describing gaming mechanics able to promote engagement through the satisfaction of the aforementioned basic needs.
Competence need 
The satisfaction of this need can be provided balancing the ratio between game difficulty and player skill in such a way that the player is placed half way between being bored and overwhelmed by the game, in other words the player needs to feel competent while they play [4]. Despite being a rightful point, the aforementioned process does not describe a motivational factor, but rather the best way through which a player can be introduced to the in-game elements that might act as motivational factors. If a game is too difficult it constitutes an entry barrier while if it’s too easy it might fail to sufficiently engage the user. Moreover, this consideration doesn’t take into account games that, by design, aims to be very difficult (e.g. triggering the need for a challenge) or very easy (e.g. heavily story driven games).
Autonomy need 	
The satisfaction of this need can be provided allowing the players to advance through different challenges and shape the game world in accordance to their will, in this way the players are allowed to act with freedom within the game.
Relatedness need 
The satisfaction of this need can be achieved though the social interactions within the game. A good example of how satisfying this need can increase the motivation to play are games having multiplayer mechanics as core features which are notoriously able to successfully engage large audiences. 
Mastery of controls
The satisfaction of this need can be provided allowing the player to master the controls of the game, this put the players in the position to experience a sense of control and receive appropriated feedbacks from the game, indeed complex mechanics are usually not enjoyed by players as they are seen as a price of admission. Again here the focus seems to be on usability rather than on defining motivational factors. Complicated controls may constitute a barrier preventing some players to experience the game while simple controls allows a larger pool of people to . If we look at some of the most engaging games available, the complexity of the controls range from very simple (e.g. swiping a finger on a screen) to extremely complex (e.g. managing combinations of multiple keys with high temporal constrains).
One of the core point in [4] is that when an activity puts pressure on achieving rewards and avoiding punishments it fosters extrinsic motivation, with potential negative effect on engagement. However, the aforementioned statement seems to be contradicted by the characteristics of some of the most captivating videogames titles, which poses a strong accent on both rewards (i.e. mobile puzzle games) and punishments (e.g. souls-like games or Multiplayer Online Battle Arenas). We won’t go into the merit if the chosen model is a good approximation for the motivational process in humans, but what we can highlight is how its application to the videogame context seems to address more usability issues rather than identifying the presence of motivational factors in games. Despite [4] made an effort for creating a bridge between psychology and the game studies, the result differently from the aforementioned models seems to ‘force’ the theoretical point of view found in [3] to a system (i.e. videogames) without knowing or taking in consideration the characteristics of this last one.
SUMMARY, PROS AND CONS
From this brief overview of some of the works trying to describe the motivational pulls of videogames, we can observe that there is no consensus over a specific taxonomy or theoretical approach. However, some common features among the various models can be observed:
1.	A motivational factor usually has the capacity to generate positive reactions in the player
2.	A motivational factor usually can be individuated in the in game activity in which a player engage the most
3.	There seems to be some common factors across various models: achievement, socialization, exploration and fighting. However this might well be an artefact derived by most of the models having a common ancestor (i.e. [38]) and being influenced by each other.
The relative incongruence among the theoretical approaches present in the literature is that each one tries to find an unified theory for describing a phenomena which is extremely prone to variability: models developed between different game genres (and even within the same genre) my result highly inconsistent due to the different characteristics of each game. Moreover, another critical point is the absence of a solid ‘anchor point’ from which the model can be developed (e.g. how in psychology there is a relatively accepted definition of the personality construct).
These theoretical approaches has a series of advantages:   
1.	They focus on videogames taking in consideration their peculiar characteristics
2.	They have been developed within the area of videogame studies, serving the purpose of providing explanatory models to researchers and professionals who works in the field.
As well as disadvantages:
1.	They are often heavily dependent on specific videogames genres
2.	They seem difficult to generalize across different games
3.	They assume a static conceptualization of motivation than differently from other psychological constructs (i.e. personality) is inherently dynamic
4.	There is very little empirical work done for validating each model

In everyday life, individuating a connection between behaviour and particular propensities is rather complicated:  an individual behaves in response to an environment that has an extremely high level of complexity [41], on the contrary the virtual world in which a videogame exists is a simplified crafted environment built for eliciting a restricted amount of behaviours [41]. In this view the tendency towards specific driving factors in a game might be described as a combination of player preferences and available sets of features that a game possesses [41].
EVALUATING THE ENVIRONMENT
As mentioned before, one the two key components for adequately assess the interaction between the player and the game is the game ‘environment’ [5] or the collection of structural characteristics proper of a specific game. Evaluating these elements is of pivotal importance since ultimately they are the only objects (or as we said before, stimuli) available to the player when interacting with the game world. 
Videogames structural characteristics
Different attempts has been made for adequately describe the in-game elements that may have an influence on the player experience. In a taxonomy proposed by [11] the authors outlined a series of features summarizing some of the most prominent videogames structural characteristics, namely: social features, manipulation (e.g. crafting) feature, narrative features, reward/punishment features and aesthetic features. In a second work by one of the authors [17] the previously proposed taxonomy was successfully employed for evaluating players based on which structural characteristics were driving their gaming behaviour. 
Reward structures in videogames
Following a different approach [9] narrowed the description of game structural characteristics reframing them as various forms of reward, like: score systems, in-game items and resources, achievements systems, feedback messages, animation events and the unlocking of new game contents. The authors also described a series of attributed that the in-game rewards should have for being effective: they need to possess social value within a shared environment, they need to have an effect on the gameplay, they need to be collectible and they need time and effort for being obtained. In completion to the aforementioned reward taxonomy, [10] focused on the description of the temporal characteristics that in-game reward may possess, they can be: limited in duration, transient and context dependent, permanent or consumable. 
Directed action model 
In an unpublished doctorate thesis by [41] the author proposed a behavioural model based on in game-actions for evaluating the direction in which the player engagement is moving. The underlying idea was that the actions performed by a player within a game could be grouped in different categories depending on their “direction” (similarly to what [38] did): social (directed towards other players), achievement (directed towards in-game goals) and fantasy (directed towards escapism), and that these directions were indicative of the motivational factors driving the engagement in the playing activity. As we can see the resulting categories resemble those individuated by some of the models previously presented [19, 39], but the approach here is based on in-game behaviour rather than relying on self-report measures. Despite being an interesting perspective and introducing a novel approach in which the player propensity can be inferred from the game behaviour, again the use of pre-defined categories fails to generalize to a wider range of games (i.e. an user evaluated in a game which doesn’t have social features won’t be able to express the propensity towards that specific engaging factor).
SUMMARY
What the works presented so far show is that each different videogame could be considered as an instantiation of a particular virtual environment with proper structural characteristics (i.e. in-game features) and rules (i.e. games mechanics). The player voluntary chose to be an individual in this ‘simulated’ environment, with supposedly no external factors forcing him to stay and behave inside it. Therefore, the question is, what drives their propensity to engage in activities inside the game world? From the work of [19, 38, 39] we saw that there are a series of motivational factors able to account for that and that this factors might be well reflected in the the structural characteristics individuated by [9, 10, 11]. At this point, we might use the theoretical framework provided early on and speculating that an individual, after voluntarily entering the game world, will interact qualitatively and quantitatively with those game features able to act as motivational factors, or in other words, with those features able to provide a pleasurable experience in that particular point in time.

\section{Measuring Motivation and Engagement}
\label{measuring_motivation_engagement}
Now we will illustrate some of the measures that can be employed for evaluating and characterizing the engagement process in videogame players. An in-depth description of this methods would be out of scope for this review so we will limit to a general description outlining advantages and disadvantages for each measuring technique.
    \subsection{Self-report Measures}
    \label{self_report}
    These measures include all those techniques requiring individuals to report their experiences and personal, psychological or demographical characteristics, they require a voluntary and cognitively elaborated answer from the individual and usually aim to measure static/stable attributes. Questionnaires require people to answer a set of questions supposedly indicative of a latent construct that an experimenter wants to measure. They are often employed because supposedly simple, easy to administer and highly scalable, moreover when compared to other kind of measures they offer the possibility to investigate a large set of constructs; that said, some important and major limitations need to be pointed out. Questionnaires often require a cognitive appraisal of actions, emotions or attitudes that despite being essential is not always accessible or able to precisely describe situations in which there is no complete awareness of the motives behind a set of actions [18]. In this view is important to say that conscious appraisal and actual experience may interact or overlap but often don’t coincide [12]. In this view it has to be reported that many attempts in correlating game behaviour and the result to psychometrics tests often produced fragmented and inconclusive findings [29]. A possible reason for this may be the nature of the employed questionnaire, not developed for assessing the characteristics of an individual within a ‘simulation’ of the real world and therefore not being able to describe the in-game behaviour [5]. Some questionnaire for addressing game specific constructs are present in the literature [19, 20] but other than carrying similar problem in explaining the player behaviour they don’t report extensive validation procedures as it is often required for the creation of psychological questionnaires. A series of other limitations affect the employment of questionnaires [21], the adherence of the respondents to social desirability norms, erroneous interpretation of the questions, untruthful answers, constrains in the possible answers, random or systematic measurement errors and in case of mass administrations (i.e. mailing or online recruiting) poor sampling control.
    \subsection{Psychophisiological Measures}
    \label{psychophisio}
    Questionnaires should be able to describe the psychological aspects of the player but are static indices prone to bias or systematic measurement error. On the other hand behavioural measures can account for external manifestations of the player experience in a more objective and naturalistic way but lack in the ability to represent the internal state of the player. As a mediation between this two poles we can pose the measurement of the physiological activity as a proxy for psychological processes (i.e. psychophysiological measures), the core assumption here is that particular physiological responses from the body are linked to basic emotions and cognitive processes [5, 23]. This kind of measures can be derived from a set of techniques ranging from more basic and general (e.g. skin conductance, Electrocardiogram) to more sophisticated and specific ones (e.g. Electroencephalogram, Functional Magnetic Resonance Imaging). The application of psychophysiological measures in a videogame context arises from the assumption that what happens in the game alters the players’ psychological state which in response may affect the body’s physiological response.  Monitoring such body alterations during a game session can assist in reconstructing the player experience [24] as well as in producing more rich players’ profiles and models [5]. Despite being powerful and precise indices for physiological manifestations, their applicability is damped by a series of limitations: they are often perceived as invasive by the individuals [5], depending on the employed hardware they might be expensive, they require particular care in the recording phase, they are extremely prone to artefacts, the signal pre-processing is often a long and laborious activity, their interpretability may be difficult if not a priori hypothesis are formulated, they often require appropriated and minimalistic experimental designs for controlling confounds and investigating only variables of interest [25].
    \subsection{Behavioural Measures}
    \label{behavioural_indices}
    Behavioural measures in a videogames context pertain mainly what can be identified as telemetries, which are high frequency records of the player behaviour within a game. The advantages of this kind of measures are: they don’t require a cognitively elaborated answer from the player (often crucial but possibly biased), given the nature of the gaming activity (usually voluntarily started) they are often high in ecological value, given the widespread diffusion of videogames they can be collected at a high frequency and on a large scale. The assumption underlying the employment of these measures is that the inputs from the players are a manifestation of their experience which can eventually be inferred analysing patterns of interaction with the game. That said, the aforementioned assumption is defined as the typical inverse problem, the player status is only indirectly inferable from the game metrics which can only be used for approaching the likelihood of the presence of certain player experience [5], with results that may work at the population level but not hold at the individual level [5]. In this view, for fully understand the player behaviour an investigation of the underlying causes becomes necessaire. However, answering the question ‘why a player behave in a particular way?’ usually needs a different approach where hypothesis testing is carried out in a restricted but controlled environment. This is the case when analytical and user oriented investigation methods are applied together [15].
    \subsection{Challenges from Large Scale Observational Data}
    \label{challenges_large_scale}
    \lorem

\section{Estimating Motivation and Predicting Engagement from Observational Data}
\label{estpred_motivation_engagement}
MODELLING APPROACHES
When trying to represent the player characteristics a distinction between profiling and modelling has to be done. Player profiling is the description of the player in a static manner that supposedly doesn’t change during the gameplay [5]. A profiling approach aims to define a restricted number of categories in which player can be divided, the characteristics of each category should be able to describe the player behaviour in a wide range of situations [5, 21, 26]. Player modelling is the study of computational models of players within a game environment. A modelling approach aims to predict the player’s experience through the evaluation of cognitive, affective and behavioural patterns arising from the dynamic interaction player-environment during gameplay [5]. In summary player modelling attempts to model the player’s playstyle while profiling aims to model the player’s characteristics. In this context two different approaches can be identified:

Model based approach (top down)		
Following this approach, in a first moment a theoretical model is hypothesized for explaining a phenomena (usually derived from a specific theoretical framework) then an empirical phase is carried out for experimentally determine if the previously hypothesized model fits the observations. Caution has to be posed in the selection of the framework in respect to its generalizability, for instance theories of motivation developed for explaining real world phenomena may not extend to a gaming context [5].

Model free approach (bottom up)		
In this other approach, observation are collected and analysed to generate models without a strong initial assumption on what the model captures. This approach assumes the presence of an unknown function between the data and the reality but does not assume anything about the structure of this function [5]. Despite being able to providing satisfying results, usually this approach provides difficult to interpreter insights on the causes behind a specific phenomenon.

Hybrid approach		
A more flexible strategy is to consider a blend of the two aforementioned approaches where insights derived from a specific theoretical framework (or from a sets of experiments) are employed for better informing the construction of a model from a bottom up perspective.

MODELLING ENGAGEMENT THROUGH TELEMETRIES
A solution for mitigating some of the limitations previously specified could be to slightly give up on specificity and abandon the idea of generating a novel theoretical framework, in favour of a practical approach aiming to generalizability and scalability. Employing some of the concepts presented before, such a solution could be identified as a hybrid behavioural model: where a complete data-driven approach is guided and mitigated by a pre-defined theoretical framework.

ASESSING THE LEVEL OF ENGAGEMENT 
Trying to quantify and describe engagement while playing a game might be a non-trivial problem: motivation per se is not directly observable or measurable, it’s rather a latent variable influencing measurable outcomes [6, 31]. In this view the challenge then becomes individuating the appropriated behavioural indicators for quantifying the level of engagement. Aiming to something similar, [6] retrieved a series of behavioural features derived from the interaction between player and hardware in a physical game in the attempt to model player’s level of involvement and enjoyment, the authors found that the most informative features for this type of tasks were those representing the frequency and strength of interaction between the player and the physical hardware. Addressing the problem from a different perspective, we can look at the decision of the player to leave a game (i.e. churn), this particular behaviour can be reframed as one of the phases of the aforementioned engagement process (i.e. disengagement), in which the player is no longer in a state able to produce gaming behaviour. In this view, various works [13, 14, 15] in the attempt to predict the players’ departure from an engaged state employed and derived particular benefit from the evaluation of meta-behavioural metrics related to the frequency and amount of gaming behaviour. What has been presented until now partially illustrate how the employment of behavioural metrics can be a useful solution for deriving proxy measures of level of engagement, in particular metrics related to frequency and amount of behaviour appear to be the most suitable. This falls in accordance to what is a common practice in behavioural sciences when it comes to quantify the amount of goal directed behaviours [16].

ASSESSING THE DIRECTION OF ENGAGEMENT	
Starting from the assumption that inter-individual differences may influence the interaction with a game and that these differences may reflect in various in-game behavioural patterns [26], an attempt can be made to solve the inverse problem of deriving a description of these differences from the analysis of behavioural data. In a work by [26] the relationship between a large number of behavioural metrics (derived from the interaction of 44 participants with a game) and the score to the NEO-PI-R (a questionnaire for the evaluation of the five-factor model of personality) was investigated. The results from this piece of work showed the presence of multiple, but a-specific, correlational patterns opening the way to the possibility of inferring self-report indices of personality from the gaming behaviour. Following this approach, in [27] the relationship between self-report measures of psychological characteristics and gaming behaviour was investigated to a larger scale and in more detail. Again, the emergence of multiple relations between specific in-game behaviours and psychological indices introduced the idea that personal characteristics may be inferred from the interaction with a videogame. In another work by [21] more control was added to the aforementioned methodologies, the authors specifically designed a short game for retrieving behavioural metrics supposedly able to mirror the construct of extraversion measured through the relative NEO-PI-R scale and sub-facets. The results again showed the emergence of correlational patterns between behaviour and questionnaire’s indices indicating the possibility to measure extraversion through the analysis of behaviour within a virtual world. Adopting a different approach [28] investigated the influence of country of origin (used as a proxy measure for the players’ cultural characteristics) on playstyle in a competitive first person shooter. Despite significant differences in the in-game behaviour between countries emerged, the estimated effect sizes were quite modest. An important caveat has to be taken in consideration in regard to these type of analysis, if the aim of the study is to derive meaningful theoretical conclusions through conventional statistical procedure, the combination of large sample sizes and multiple tests (i.e. dozens or hundreds of correlational analysis) is likely to provide significant but often spurious and difficult to interpreter results. On the other hand, if the aim is to infer the results to psychological questionnaires from the gaming behaviour, the use of analysis evaluating one to one, many to one or many to many linear relationships may be not appropriated, this may be reflected in to the fragmented and sometimes counter-intuitive findings relating behaviour and psychological indices [29]. What we propose here is to partially step back from the individuation of a linkage between self-report indices of personal characteristics and behaviour focusing exclusively on a descriptive methodology of the latter. Adopting an approach similar to [41] the aim would be to evaluate how much playing activity an user dedicate to a particular game features, providing this way a proxy measure of the direction of what [40] defined as period of engagement.

STATIC OR DYNAMIC?
When trying to develop a methodology for assessing, predicting or modelling the player’s engagement, a point has to be taken in mind: individuals (i.e. players) are not monad, they differ in the type of games they enjoy, in the way they interact with the gaming environment and ultimately in the type of features they prefer to interact with [12]. Therefore, for understanding how and why players behave differently inside a game we have to understand their underlying motives [12] and their reception of the game’s engagement driving mechanisms [3]. A simplified (although practical) solution to the problem would see the players grouped into limited types based on their personal characteristics and attitudes towards the game [3]. This can be defined as the stereotype approach, which automatically assign a player to a pre-defined group for which the key characteristics have been previously defined [5]. But a strong caveat can be found with this approach: the propensity for a specific in-game element may vary over time or being influenced by situational factors [7]. In this view, the adoption of a model able to account for the dynamic nature of the interaction player game-environment could provide a more flexible tool for describing the motivational process.

\subsection{Latent States Estimation Models}
\label{latent_estimation}
\lorem
\subsection{Engagement Prediction Models}
\label{engagement_prediction}
Given space restrictions, it is not possible to exhaustively describe all the works related to survival and churn estimation, we will therefore focus on some key examples closely linked to our work. When it comes to estimating churn, in particular for industry applications, it is relevant to develop and test models considering different titles and genre. In this view \cite{runge2014churn, hadiji2014predicting, xie2015predicting, kim2017churn} are notable examples where a range of different modelling techniques (i.e. Linear Models, Decision Tress, Naive Bayes, Support Vector Machines and Deep Neural Networks (DNN)) were tested on a churn estimation task across multiple game titles. However, they often employed game specific features (sometimes carefully engineered) and build and test separate models for each game. A notable exception to that is the work done by \cite{liu2018semi}, where churn was formalized as edge prediction in a dynamic graph and modelled through a DNN. This produced a single model able to generalize across multiple game titles however with the limitation of them being all mobile titles. Another important characteristics for a churn estimation model, is to be able to produce predictions even when minimal observations are provided. In this view the works done by \cite{drachen2016rapid, milovsevic2017early} highlights the effort made in the literature for designing methodologies able to rapidly provide estimations of churn probability via minimal amount of metrics, this was done employing traditional machine learning algorithms (same as above) but exclusively considering metrics recorded during the initial stages of the user-game interaction. One of the major drawbacks in these works was that the initial period of observation was arbitrarily chosen and fixed for all the considered users making it difficult to take inter individual differences into account. In regard to the literature on survival analysis, we found that most works employed Cox Regression \cite{cox1972regression}, or some variation of it, for estimating the probability to survive (i.e. not have churned) after a specific period of time \cite{perianez2016churn, bertens2017games, demediuk2018player}. Despite being a similar formulation, this is not equivalent to estimating the survival time (i.e. the amount of future playing time), which becomes much more interesting when trying to assess not only measures of disengagement but also measures of future sustained engagement. A notable exception to this is the churn and survival analysis competition presented by \cite{liu2018semi}, where the goal was to estimate both churn probability and survival time, this also highlights the growing interest for richer assessments of user engagement and for models able to perform both tasks. On top of what is illustrated so far, we also individuated a series of limitations regarding the employed data-sources. The number of the considered users rarely goes beyond $10^4$ \cite{liu2018semi} and when it comes to churn estimation the class distribution is usually greatly imbalanced, both of these factors can pose limitations on the interpretation and generalization of results. 


In this section we give a general overview on the state of the art in engagement modelling. Due to space constraints and the substantial literature on the matter we will focus on a restricted set of representative works in the area of large-scale behavioural modelling of engagement. The work on engagement modelling comes, generally, in two forms: estimation and profiling of in-game behaviour \cite{el2016game}. The estimation of in-game behaviour is usually formulated as a supervised machine-learning or more general statistical modelling problem\cite{el2016game}. Despite the literature on the topic often presenting compelling solutions for practical problems, it tends to follow a black box approach: a machine-learned solution is generated for solving a specific task but no attempts are made to inspect or interpret the model \cite{lee2018game, liu2019micro, del2020time, kristensen2019combining}. Moreover, when these attempts are made, the lack of a solid and predefined theoretical framework tends to lead to post-hoc interpretations which are sometimes difficult to verify or relate with actual human behaviour \cite{drachen2016rapid, del2019profiling}. When trying to estimate engagement profiles, the approach widely used in the literature is to adopt some form of unsupervised learning technique for individuating patterns of interaction with various in-game features \cite{el2016game, del2019profiling}. This however is usually done considering an unconstrained set of game-specific metrics. As a result, \textit{a-posteriori} justifications for the characteristics of the individuated profiles are provided \cite{drachen2012guns, makarovych2018like, drachen2009player}, which, without an overarching explanatory theoretical framework, appear to be be very context-specific and difficult to interpret. What we see in the literature is that attempts are made to model a single behavioural manifestation of engagement rather than the construct in its entirety. A noticeable exception in this regard is the recent work by Reguera et al. \cite{reguera2020quantifying}, who adopt a complete data-driven approach managed to derive a general law for describing and quantifying the engagement process, similarly to what Bauckhage at all. did in \cite{bauckhage2012players}. However, neither group interpret their findings through the lens of existing human behaviour theory. We believe that a holistic model of engagement can be generated, constraining the great flexibility provided by data-driven approaches by employing solid and well established theoretical priors \cite{yannakakis2013player}. To do so, an \textit{a-priori} theoretical framework which is guaranteed to generalise to different situations should be defined. Such a framework should clearly state what are the observable and measurable indicators of engagement and how they are expected to vary in relations with the construct's dynamics. In doing so the findings emerging from data-driven approaches can be compared with what the theoretical framework prescribes.
\subsection{Common Features}
\label{estimation_prediction}
\lorem

\lorem



