\chapter{Ancillary Model Performance Analyses}

\section{Bayesian Multi-Level Model Definition}
\label{bayesian_multilevel_model}
In order to assess in a more robust and reliable way the results of our models comparison experiments, along with the conventional frequentist analyses presented in chapter \ref{chapter_implementation_testing} we carried out an additional set of statistical analyses using a bespoke Bayesian multi-level model. We opted for a time-varying random intercept accounting for the impact of game context and time and a fixed slope for estimating the effect of model. We adopted partial pooling strategy for the time-varying random intercept as we could safely assume exchangeability of the various game contexts \cite{gelman2020bayesian}. The model had the following formulation:
\begin{gather}
    \label{bayesian_mlm}
    SMAPE_{i, j, t} \sim StudentT(\nu, \mu, \sigma) \\ \nonumber
    \nu \sim Gamma(\alpha=2, \beta=0.1) \\ \nonumber
    \sigma \sim HalfCaucy(\beta=1)  \\ \nonumber
    \mu = \beta_{jt} + \alpha_{i}  \\ \nonumber
    \alpha_{i} \sim Normal(\mu=0, \sigma=1)  \\ \nonumber
    \beta_{jt} = \alpha_{j}^\top B_{t,*} \\ \nonumber
    \alpha_{j} \sim Normal(\mu_j, \sigma_j) \\ \nonumber
    \mu_j \sim Normal(\mu=1, \sigma=1) \\ \nonumber
    \sigma_j \sim HalfCauchy(\beta=1) \\ \nonumber
\end{gather}
with $B \in \mathbb{R}^{T \times dof}$ being a B-spline matrix with $dof=6$ as provided by the python library Patsy \cite{patsy}. In this case $j$ indicates the $j^{\text{th}}$ game context, $i$ the $i^{\text{th}}$ model and $t$ the $t^{\text{th}}$ element in the considered sequence of SMAPE values. The $StudentT$ likelyhood allowed for an estimation of $\mu$ robust to outliers. The values for the the $Gamma$ distribution parameters were chosen according to a sensible default as mentioned in \cite{vehtarinu}. The choice of priors for $\alpha_{i}$ were made so to have a small regularizing effect: the priors assume each model to have an impact on the base SMAPE of $\sim \pm 2\% $ but with the expectancy of having no effect. The same model was fit on each of the 5 targets separately using mean field approximation \cite{kucukelbir2017automatic}. The model was implementing using the python library PyMC3 \cite{salvatier2016probabilistic}. As mean field approximation doesn't allow conventional convergence checks for Bayesian model fitting, we relied exclusively on visual inspection of the trace plot and marginal probability distributions. Comparison between different models was carried out by inspecting the marginal distribution of differences in the $\mu$ parameter, highlighting a region of practical significance (ROPE) equals to $\{-0.1, 0.1\}$. This means that we deemed worth of attention only differences in $\mu$ greater than $0.1\%$ SMAPE.

\subsection{Dynamic Prediction of Future Behavioural Intensity}
\label{dynamic_prediction_ancillary_perf}

\subsection{Dynamic Prediction of Future Behavioural Intensity with Environmental and Game Covariates}
\label{dynamic_env_even_prediction_ancillary_perf}

